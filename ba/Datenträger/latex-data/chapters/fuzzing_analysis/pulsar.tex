%! Author = chaorn
%! Date = 13.08.24

\subsection{Pulsar}\label{subsec:pulsar}
Im Folgenden werden alle Schritte zur Vorbereitung und Durchführung der Fuzzing Kampagne mit Pulsar beschrieben und
anschließend der daraus entstandene Erkenntnisgewinn geschlussfolgert.
\subsubsection{Setup}
Pulsar wurde mithilfe des von Pahl bereitgestellten Docker-Images~\cite{pulsar-docker} installiert.
Das Docker-Image wurde auf einem Ubuntu 20.04 LTS System installiert.
Weitere Abhängigkeiten, welche nicht in der Installation der Docker-Images enthalten sind, wurden manuell installiert und
in einer \texttt{requirements.txt} Datei festgehalten.
Für das Fuzzing mit Pulsar wird ein Netzwerkverkehr benötigt.
Dieser Netzwerkverkehr wird in Form von \gls{pcap}-Dateien gespeichert.
Hierfür wurde ein Python-Skript entwickelt, welches \gls{mqtt} spezifische Nachrichten an den Broker sendet und möglichst
viele Zustände vom Broker erreicht.
Die Nachrichten werden mithilfe des Tools \texttt{tcpdump} aufgezeichnet und in \gls{pcap}-Dateien abgelegt.
Die \gls{pcap}-Dateien werden in einem Ordner \textit{traffic} abgelegt.
Für den Start werden zusätzlich die \gls{pcap}-Dateien im Ordner \textit{traffic} über ein live-volume in den
Docker-Container eingehängt.
Alle gesammelten pcap-Dateien dienen für den weiteren Verlauf der Untersuchung.
Pulsar verwendet die \gls{pcap}-Dateien, um die State Machine zu erlernen.\newline\newline
Die Daten wurden in Kombination eines Python-Skripts, der Fuzzing Kampagne mit \gls{afl}Net und dem Tool
\texttt{tcpdump} gesammelt.
Das Python-Skript~\cite{python-script-input-generation} und \gls{afl}Net~\cite{aflnet-capture-traffic} wurden im Falle der Vorbereitung
auf die Fuzzing Kampagne mit Pulsar als Input-Generatoren verwendet.
Mithilfe dieser Tools wurde der Netzwerkverkehr aufgezeichnet und in \gls{pcap}-Dateien abgelegt.
Die \gls{pcap}-Dateien wurden in den Docker-Container mithilfe eines live-volumes eingehängt und Pulsar wurde gestartet~\cite{run-pulsar}.
Die gesammelten Daten wurden dann von Pulsar als Trainingsdaten verwendet, um das zu analysierende Protokoll zu erlernen.
\subsubsection{Erkenntnisgewinn}
Die Erkenntnisse des Fuzzing Kampagne mit Pulsar belaufen sich dabei auf die Art und Weise, wie Pulsar
den Netzwerkverkehr analysiert und das Protokoll erlernt.
Bei der Generierung von Input mit \gls{afl}Net und dem Python-Skript wurde darauf geachtet, dass möglichst viele Zustände
erreicht werden.
Jedoch wurde bei der Analyse des Netzwerkverkehrs festgestellt, dass Pulsar nicht in der Lage war, die State Machine
des \gls{mqtt} Protokolls korrekt zu erlernen.
Die Ursache hierfür ist, dass Pulsar die Startsequenz des Protokolls nicht korrekt erlernen konnte.
Durch das Generieren von Input mit \gls{afl}Net wurden viele falsche Pakete generiert, welche auch häufiger auftraten.
Mit unter ist das der Fall gewesen, dass Pulsar eine oft vorkommende Sequenz von vielen \textit{A}s als Startsequenz
des Protokolls interpretiert hat.
Diese Sequenzen sollten einen Fehlerzustand im Protokoll darstellen und nicht als Startsequenz interpretiert werden.
Da jedoch viele Implementierungen von Fehlerbehebungen in der State Machine des Protokolls vorhanden sind, wurde von
\gls{afl}Net auch eine Vielzahl von Fehlerzuständen generiert.
Dies liegt an der Art, wie \gls{afl}Net geeignete eingaben bewertet.
Eine Eingabe, die von \gls{afl}Net als geeignet bewertet wird, wird auch häufiger generiert.
Diese geeigneten Eingaben werden als geeignet bewertet, wenn sie viele Zustände erreichen.
Da jedoch viele Fehlerzustände in der State Machine des Protokolls vorhanden sind, werden auch viele Fehlerzustände
generiert.
Die vorhergehenden Versuche eines Verbindungsaufbaus mit einer Connect-Nachricht wurden nicht als Startsequenz
interpretiert.
Dies hatte zur Folge, dass im Zeitrahmen dieser Arbeit Pulsar nicht effektiv zum Laufen gebracht werden konnte.
Pulsar benötigt nämlich viel Zeit eine Markov-Kette zu erlernen und die State Machine zu approximieren, wenn besonders
viel Netzwerkverkehr aufgezeichnet wurde.
Im Rahmen dieser Arbeit wurden bis zu 7 Gigabyte an Netzwerkverkehr aufgezeichnet und Pulsar benötigte mehrere (ca.\ 35)
Stunden, um die State Machine zu erlernen.\newline\newline
Eine weitere Erkenntnis ist -- aufgrund der Geschwindigkeit des Lernprozesses -- dass Pulsar noch keine Hardwarebeschleunigung
mit bspw.\ einer \gls{gpu} unterstützt.
Die Hardwarebeschleunigung würde den Lernprozess beschleunigen und somit auch die State Machine schneller erlernen.