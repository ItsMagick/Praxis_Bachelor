\section{Einleitung}\label{sec:einleitung}
In der heutigen digital vernetzten Welt sind Netzwerke und ihre Protokolle das Rückgrat zahlreicher kritischer Anwendungen,
von einfachen Webdiensten bis hin zu komplexen \gls{iot}-Systemen.
Eine entscheidende Komponente der Netzwerksicherheit ist das Testen und die Absicherung dieser Protokolle gegen potenzielle Schwachstellen.
Hierbei kommt die Technik des Fuzzing zum Einsatz, die durch die gezielte Überlastung eines Systems mit fehlerhaften oder
unerwarteten Eingaben mögliche Sicherheitslücken aufdecken soll.
Besonders relevant ist die Analyse von Netzwerkfuzzern, die speziell für die Sicherheitstests von Netzwerkprotokollen entwickelt wurden.
Das \gls{mqtt}-Protokoll, ein leichtgewichtiges Messaging-Protokoll für kleine Sensoren und mobile Geräte, stellt in vielen
\gls{iot}- und \gls{m2m}-Anwendungen einen wichtigen Standard dar.
Aufgrund seiner weitreichenden Nutzung und der oft sensiblen Natur der übertragenen Daten, ist es essenziell, die Sicherheit
dieses Protokolls gründlich zu prüfen.
Netzwerkfuzzer wie Pulsar, \gls{afl}Net und boofuzz bieten unterschiedliche Ansätze zur Durchführung solcher Tests.
Pulsar fokussiert sich auf eine dynamische und effektive Fehlersuche durch einen Mix aus Netzwerk- und Protokoll-basiertem
Fuzzing.
\gls{afl}Net basiert auf dem bewährten Ansatz von \gls{afl} und passt diesen für Netzwerkprotokolle an, um eine effektive
Testabdeckung zu gewährleisten.
Boofuzz -- eine Weiterentwicklung des etablierten Sulley-Fuzzers -- bietet spezifische Funktionen für das Testen von
Netzwerkdiensten und deren Protokollen.
In dieser Arbeit wird die Performance dieser Netzwerkfuzzer im Kontext des \gls{mqtt}-Protokolls detailliert analysiert.
Ziel ist es, durch einen Vergleich der Fuzzing-Techniken und deren Effektivität Erkenntnisse zu gewinnen, die über die bloße
Anwendung hinausgehen und zu einer fundierten Bewertung der getesteten Tools führen.
Die Untersuchung fokussiert sich auf folgende Forschungsfragen:
\begin{questions}\label{researc-questions}
    \item Welche Performanceunterschiede gibt es unter den Netzwerkfuzzern in den Aspekten von gefundenen Bugs, Ausführgeschwindigkeit
        und Effizienz der generierten Eingaben?
    \item Sind Fuzzer in der Lage künstlich platzierte Schwachstellen in Netzwerkprotokollen zu identifizieren?
\end{questions}
Performance soll hierbei aus den Metriken der Anzahl der gefundenen Bugs, der Ausführungsgeschwindigkeit und der Effizienz
der generierten Testfälle gemessen werden.
Die Ergebnisse der Tests werden in Kapitel~\ref{subsec:vergleich-der-erhobenen-metriken} vorgestellt und diskutiert.
Hierbei wird die Forschungsfrage \textit{Q1} beantwortet.
Die Forschungsfrage \textit{Q2} wird in Kapitel~\ref{sec:conclusion} beantwortet und diskutiert.