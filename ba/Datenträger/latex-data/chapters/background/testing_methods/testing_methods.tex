%! Author = charon
%! Date = 6/4/24

\subsection{Software-Testing Methoden}\label{subsec:software-testing-methoden}
Das Fuzzing ist eine Software-Testing Methode zum Finden von Bugs und Sicherheitslücken in Applikationen.
Es gibt verschiedene Arten von Software-Testing Methoden, die in der Software-Entwicklung eingesetzt werden.
Die drei Kategorien sind White-Box, Black-Box und Grey-Box Testing.
Sie unterscheiden sich in der Art und Weise, wie sie die Software testen.
Jede Methode hat ihre eigenen Vor- und Nachteile und wird in verschiedenen Situationen eingesetzt.
In den folgenden Abschnitten werden die verschiedenen Software-Testing Methoden genauer erläutert.
\subsubsection{White-Box testing}\label{subsubsec:white-box-testing}
Es handelt sich um White-Box Testing, wenn der Tester Zugriff auf den Quellcode der Software hat.
Diese Testingmethode wird auch als \textit{Clear-Box}, \textit{Transparent-Box} oder \textit{Open-Box} Testing bezeichnet.
Der Tester kann den Quellcode der Software analysieren und die Testfälle basierend auf dem Quellcode erstellen~\cite{white-box-testing}.
White-Box Testing wird normalerweise von Entwicklern durchgeführt, um sicherzustellen, dass der Code korrekt funktioniert und keine Fehler enthält.
Es wird auch verwendet, um die Testabdeckung zu messen und sicherzustellen, dass alle Teile des Codes getestet wurden.
White-Box Testing ist eine sehr effektive Methode, um Fehler in der Software zu finden,
aber es erfordert auch eine gründliche Kenntnis des Quellcodes und der Software-Architektur.
Zu White-Box Testing gehören mitunter das \textit{Unit Testing}, bei dem einzelne Komponenten der Software getestet werden
und \textit{Path Testing} bei dem alle möglichen Programmpfade innerhalb eines zu testenden Abschnittes abgelaufen und überprüft werden sollen.\newline
Der Vorteil dieser Testmethode besteht darin, dass hierbei komplexe und auch strukturelle Fehler aufgedeckt werden können.
Die Anwendung einer solchen Testmethode findet im optimalen Fall während der Entwicklung einer Anwendung statt.
\subsubsection{Black-Box testing}\label{subsubsec:black-box-testing}
Bei Black-Box Testing wird kein Zugriff auf den Quellcode und die interne Struktur der Application gebraucht.
Das Ziel dieser Testmethode ist es zunächst mögliche Eingaben von Benutzern einer Software nachzuahmen~\cite{black-box-testing}.
Somit werden funktionalitäten einer Software getestet, ohne dass der Tester den Quellcode kennt.
In der Regel werden hierbei nur bereits kompilierte Software-Module oder von außen Zugreifbare Endpunkte getestet.
Anwendungsfälle einer solchen Herangehensweise entsprechen einem Penetrationstest oder eines Usability-Tests.
Der Vorteil von Black-Box Testing besteht darin, dass das \gls{zup} aus der Sicht des Answers getestet wird.
Dabei wird die Software auf ihre Funktionalität und Benutzerfreundlichkeit getestet.
\newline
Black-Box Testing ist eine effektive Methode, um sicherzustellen, dass die Software den gestellten Anforderungen
entspricht und keine offensichtlichen Fehler enthält.
Es ist jedoch schwierig, komplexe Fehler zu finden, da der Tester keinen Zugriff auf den Quellcode hat und nicht weiß,
wie die Software intern funktioniert.
\subsubsection{Grey-Box testing}\label{subsubsec:grey-box-testing}
Grey-Box Testing kombiniert beide Aspekte der bereits genannten Ansätze.
Hierbei werden Teile des \gls{zup} offengelegt.
Während beim White-Box Testing die interne Struktur oder der Code des Systems bekannt und getestet wird und beim Black-Box
Testing nur die externen Funktionalitäten ohne Wissen über den internen Aufbau geprüft werden,
ermöglicht Grey-Box Testing eine Balance zwischen diesen Ansätzen.
Beim Grey-Box Testing hat der Tester teilweise Kenntnisse über die interne Struktur der Anwendung oder des Systems,
verwendet diese Informationen jedoch hauptsächlich, um die Erstellung und Durchführung von Testszenarien zu optimieren,
die auf der Benutzeroberfläche basieren~\cite{Coulter2001GrayboxST}.
Diese Methode erlaubt es, gezielte Testfälle zu entwerfen, die nicht nur auf den sichtbaren Eingaben und Ausgaben basieren,
sondern auch auf dem Wissen über die zugrundeliegenden Datenflüsse, Algorithmen oder architektonischen Schwächen.
Grey-Box Testing wird häufig in Szenarien eingesetzt, in denen eine vollständige Kenntnis der internen Struktur nicht
notwendig oder verfügbar ist, jedoch ein gewisses Maß an Verständnis über die Architektur oder den Code notwendig ist,
um tiefere Testfälle zu entwickeln.
Diese Technik kann besonders nützlich sein bei der Validierung von Sicherheitsaspekten, der Integrationstests von komplexen
Systemen oder der Optimierung der Testabdeckung bei großen Anwendungen.
In der Praxis findet Grey-Box Testing Anwendung in Bereichen wie Web-Applikationen, APIs und bei sicherheitskritischen
Systemen, bei denen sowohl die Funktionalität als auch die interne Verarbeitung zuverlässig und sicher getestet werden müssen.
Die Methode bietet somit eine ausgewogene Teststrategie, die die Vorteile von White-Box und Black-Box Testing vereint,
um eine effektivere Fehlererkennung zu ermöglichen.