%! Author = charon
%! Date = 5/23/24
\subsubsection{Generierung von Input}\label{subsubsec:generierung-von-input}
Die Generierung von Input ist ein wichtiger Bestandteil des Fuzzing-Prozesses.
Hierbei unterscheiden sich die Fuzzer in ihrer Herangehensweise und den Techniken, die sie verwenden, um Eingaben zu generieren.
Die Generierung von Input kann auf verschiedene Arten erfolgen, die in den folgenden Abschnitten näher erläutert werden.\newline\newline
\textit{Generationsbasierende}~\cite{iot-fuzzing} Fuzzer generieren Eingaben, indem sie eine Menge von Eingaben in Epochen erstellen.
Jede Epoche wird als Generation bezeichnet.
Diese Eingaben basieren auf einem vom Anwender festgelegten Regelwerk.
Diese Herangehensweise des Generierens von Eingaben ist effektiv, um die Software mit Eingaben zu versorgen, welche nicht
bereits im Vorfeld von dem \gls{zup} als fehlerhaft befunden werden sollen.
Beispielsweise ist dies Vorteilhaft, wenn besonders komplexe Syntax des Input vonnöten ist.\newline\newline
\textit{Mutationsbasierende}~\cite{iot-fuzzing} Fuzzer generieren Eingaben, indem sie vorhandene Eingaben, welche bereits im Vorfeld definiert wurden, verändern.
Hierbei wird der Aufbau und die Struktur der Eingaben standardmäßig nicht berücksichtigt.
Diese Veränderung wird Mutation genannt und kann auf verschiedene Arten erfolgen.
Die Mutation kann beispielsweise durch das Hinzufügen, Entfernen oder Verändern von Bytes in der Eingabe erfolgen.
Um die Güte der neu generierten Eingaben zu bestimmen, wird die Code Coverage -- also die im Quellcode erreichte Tiefe --
analysiert.
Eingaben, die besonders viele Codesegmente erreichen werden als besonders gut gewertet und als bevorzugte Eingabe für
weitere Mutationen verwendet.
