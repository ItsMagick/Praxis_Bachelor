%! Author = charon
%! Date = 6/4/24

\section{Verwandte Arbeiten}\label{sec:related-work}
Die Evaluierung und Analyse der Performance von Fuzzern ist ein Forschungsgebiet in der Cybersicherheitsforschung.
In der vorliegenden Sektion erfolgt eine Darstellung relevanter wissenschaftlicher Arbeiten und Methodiken zur Performance-Analyse
von Fuzzern sowie eine Erörterung moderner Ansätze zum Fuzzing und zur Analyse von Netzwerkprotokollen.\newline\newline
Die Forschung hat sich in zwei Hauptrichtungen entwickelt: Zum einen wird Fuzzing auf eine Vielzahl von Zielsystemen angewendet,
von großen verteilten Anwendungen bis hin zu eingebetteten Geräten~\cite{iot-fuzzing,fuzzing-assessment,embedded-fuzzing,iot-fuzzer,libafl}.
Zum anderen wurden die internen Abläufe von Fuzzern kontinuierlich verbessert, um eine effizientere Entdeckung von Schwachstellen
zu ermöglichen~\cite{fuzzing-evaluation,a-survey-of-network-protocol-fuzzing,AFLplusplus-Woot20, mutuation-analysis}.
In den letzten sechs Jahren wurden über 280 wissenschaftliche Arbeiten zu Fuzzing veröffentlicht, die zahlreiche Verbesserungen
und neue Techniken vorschlagen~\cite{fuzzing-evaluation}.\newline
Darüber hinaus wird in der Forschung die Anwendung von Mutationsanalysen zur Bewertung von Fuzzern untersucht~\cite{mutuation-analysis}.
Diese Technik ermöglicht es, Fuzzer anhand einer Vielzahl von unvoreingenommenen Mutationen zu vergleichen, um deren Fähigkeit
zur Fehlererkennung zu bewerten.
Studien zeigen, dass heutige Fuzzer nur einen kleinen Prozentsatz der Mutationen erkennen können, was eine Herausforderung
für zukünftige Forschungsarbeiten darstellt~\cite{mutuation-analysis}.\newline
Ein innovativer Ansatz in diesem Bereich ist das \enquote{Smart Fuzzing}, das maschinelles Lernen nutzt, um Test-Suiten für
Netzwerkangriffe zu konstruieren~\cite{smart-fuzzing,pulsar,cnn}.
Dieser Ansatz hat sich als effektiv erwiesen, um verschiedene unsichere Zustände in Cyber-Physical Systems zu identifizieren,
und zeigt das Potenzial, neue Angriffe zu entdecken, die in herkömmlichen Benchmarks nicht erfasst werden.\newline
Ein weiterer Aspekt ist das Reverse Engineering von Netzwerkprotokollen, das als automatisierter Prozess zur
Extraktion von Protokollformat, Syntax und Semantik definiert ist~\cite{cnn}.
Dieser Prozess erfolgt durch die Überwachung und Analyse der Eingaben und Ausgaben der Protokollsoftware, ohne auf die
Protokollspezifikation angewiesen zu sein~\cite{a-survey-of-network-protocol-fuzzing}.
Dies ermöglicht eine tiefere Einsicht in die Funktionsweise von Protokollen, insbesondere bei nicht dokumentierten oder
proprietären Protokollen.\newline
Benchmarking und Auswertung von Fuzzern sind entscheidende Aspekte der Fuzzing-Forschung, um die Effektivität und Effizienz
verschiedener Fuzzing-Methoden zu vergleichen.
Bekannte Benchmarking-Tools sind FuzzBench~\cite{fuzzbench}, das von Google entwickelt wurde und eine standardisierte Umgebung für die
Bewertung von Fuzzern bietet oder LAVA~\cite{lava}.
Es ermöglicht die Evaluierung von Fuzzern anhand verschiedener Zielprogramme und bietet wichtige Erkenntnisse über die Leistung
von Fuzzern, insbesondere in Bezug auf die Anzahl der verwendeten Anfangsseed-Daten und die Verwendung eines gesättigten Korpus~\cite{fuzzing-evaluation}.