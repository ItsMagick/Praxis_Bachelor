%! Author = charon
%! Date = 4/12/24
\clearpage
\newgeometry{top=2.4cm, bottom=2.4cm}
\pagestyle{empty}
\section*{Abstract}
Die Bachelorarbeit \textit{Auswertung von Netzwerkfuzzern am Beispiel des MQTT Protokolls} untersucht die Effektivität und Leistungsfähigkeit von drei spezifischen Netzwerkfuzzern—Pulsar, AFLNet
und boofuzz—im Kontext des MQTT-Protokolls, das weit verbreitet in IoT- und M2M-Anwendungen zum Einsatz kommt.
Netzwerkfuzzing ist eine Methode des Softwaretestens, die darauf abzielt, Sicherheitslücken und Schwachstellen in
Netzwerkprotokollen aufzudecken, indem gezielt fehlerhafte oder unerwartete Eingaben an ein System gesendet werden.
Die Arbeit beginnt mit einer umfassenden Einführung in die Grundlagen des Softwaretestens und des Fuzzings, wobei
verschiedene Testmethoden wie White-Box, Black-Box und Grey-Box Testing erläutert werden.
Es folgt eine detaillierte Darstellung des MQTT-Protokolls und der spezifischen Anforderungen, die es an Fuzzing-Tools stellt.\newline\newline
Im Kern der Arbeit steht die Analyse der drei Fuzzer, die hinsichtlich ihrer Fähigkeit, Sicherheitslücken im MQTT-Broker
Mosquitto zu identifizieren, verglichen werden.
Dazu wurden verschiedene Metriken herangezogen, darunter die Anzahl gefundener Bugs, die Geschwindigkeit der Testfallgenerierung
und die Reproduzierbarkeit der entdeckten Schwachstellen.
AFLNet, ein Grey-Box-Fuzzer, der speziell für Netzwerkprotokolle entwickelt wurde, zeigte in der Untersuchung eine hohe
Effizienz bei der Entdeckung von Schwachstellen, insbesondere bei der Analyse von Verbindungsfehlern und Buffer Overflows.
Boofuzz, ein Black-Box-Fuzzer, der durch Zufall generierte Testfälle verwendet, zeigte eine geringere Effizienz, konnte
jedoch durch seine einfache Handhabung und Flexibilität punkten.
Pulsar, das sich durch einen feedback-orientierten Ansatz auszeichnet, bot eine ausgeglichene Performance zwischen Effektivität
und Geschwindigkeit, erwies sich jedoch als weniger effektiv in der Reproduzierbarkeit von Bugs.\newline\newline
Die Arbeit schließt mit einem Vergleich der Stärken und Schwächen der getesteten Fuzzer und bietet Empfehlungen für deren
Einsatz in der Praxis.
Zudem werden Ansätze für zukünftige Arbeiten aufgezeigt, die darauf abzielen, die Fuzzing-Techniken weiter zu optimieren
und ihre Anwendung in sicherheitskritischen Netzwerksystemen zu verbessern.
Die Ergebnisse dieser Bachelorarbeit liefern wertvolle Erkenntnisse für die Weiterentwicklung von Fuzzing-Kampagnen und tragen
zur Verbesserung der Sicherheitsüberprüfung von Netzwerkprotokollen wie MQTT bei.
\restoregeometry