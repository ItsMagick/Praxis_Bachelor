%! Author = charon
%! Date = 6/4/24

\section{Konklusion und zukünftige Arbeiten}\label{sec:conclusion}
Abschließend ist zu sagen, dass die Effektivität eines Fuzzers nicht anhand der Anzahl der generierten Eingaben pro Sekunde
gemessen werden kann.
Bei komplexeren Protokollen wie \gls{mqtt} ist es wichtig, dass die generierten Eingaben valide sind und möglichst viele
Zustände des \gls{zup} erreichen.
Ähnliche Schlussfolgerungen wurden bereits in der Arbeit von Holz et.\ al.\ in \textit{\enquote{SoK: Prudent Evaluation Practices for Fuzzing}}
\cite{fuzzing-evaluation} gezogen.
Bei einem Fuzzer wie boofuzz, welcher eine Vielzahl von Eingaben generiert, ist es wichtig, dass die generierten Eingaben
strenger definiert werden müssen und nicht nur auf dem Prinzip von Brute-Force basieren sollten.
Im Falle von Pulsar ist es wichtig, dass der Fuzzer die Startsequenz des Protokolls korrekt erlernen kann, um ein Programm
effektiv testen zu können.
Auch wenn Pulsar mithilfe vieler Testdaten die State Machine des Protokolls erlernen kann, ist es wichtig auf den gesendeten
Netzwerkverkehr zu achten.
Geeignete Testdaten sollten so generiert werden, dass möglichst viele Zustände des \gls{zup} erreicht werden und möglichst
viele valide Eingaben generiert werden, welche der Fuzzer Pulsar als Anhaltspunkt zum Erlernen valider Eingaben benötigt.\newline\newline
Trotz der Tatsache, dass AFLNet in der Lage war, einige Fehler in der manipulierten Implementierung von \gls{mqtt} zu finden, war der Fuzzer
nicht in der Lage, alle Fehler zu finden.
Dies zeigt, dass die Implementierung von Protokollen und Netzwerkdiensten weiterhin fehleranfällig ist und dass
Fuzzing Kampagnen auf einen größeren Zeitraum mit mehreren Iterationen durchgeführt werden sollten, um komplexere Schwachstellen
zu finden.
Die Schwachstellen, die von \gls{afl}Net nicht gefunden wurden belaufen sich auf den Buffer Overflow in der Verarbeitung
des Verbindungsaufbaus mit dem Broker, den Off-by-One Fehler in der Verarbeitung von Publish Nachrichten und den Use-After-Free
Fehler in der Verarbeitung von Subscribe Nachrichten.
Die Forschungsfrage \textit{Q2} (siehe Aufzählung~\ref{researc-questions}) konnte somit nur teilweise beantwortet werden,
da sowohl boofuzz als auch Pulsar keine potenziellen Schwachstellen entdecken konnten und \gls{afl}Net nur drei von sechs
entdeckt hat.
Für bessere Ergebnisse müssen die Eingaben und die Implementierungen der Kampagnen weiter untersucht werden und weitere
Experimente durchgeführt werden.\newline
Zudem wurde mit der Rechnung in~\ref{subsec:boofuzz} eine Möglichkeit gegeben die Anzahl der benötigten Zeit der Generierung
eines validen Pakets gegeben.
Aufgrund der gezeigten Berechnung ist es ebenso ratsam mehrere Instanzen des boofuzz Fuzzers zu starten, die bestimmte Felder
eines \gls{mqtt}-Pakets zu fuzzen und jeder Fuzzer-Instanz ein genau definiertes Feld zuzuweisen.\newline\newline
In Zukunft sollte die Fuzzing Kampagne mit Pulsar weitergeführt werden.
Hierbei sollte darauf geachtet werden, dass fürs Erste die Startsequenz des Protokolls korrekt erlernt wird, sodass Pulsar
einen validen Startpunkt für eine effektivere Fuzzing Kampagne hat.
Außerdem werden in Zukunft weitere Experimente und Fuzzing Kampagnen mit boofuzz durchgeführt, um die Effektivität des
Fuzzers besser zu verstehen und zu testen.
Auf lange Frist soll das automatisierte Testen und Lernen von Protokollen ohne jedwede menschliche Interaktion und
Vorkenntnisse möglich sein.
Eine eigene Implementierung einer kontinuierlich lernenden Pipeline für das automatisierte Testen von Protokollen wäre
ein monumentales Ziel, welches in Zukunft erreicht werden sollte.
Diese Pipeline sollte in der Lage sein, Protokolle anhand von bspw.\ Brute-Force ansätzen zu testen und zu lernen, um
daraufhin valide Eingaben zu generieren und somit die State Machine des Protokolls möglichst genau approximieren zu können.
Diese Entwicklungen könnten in Zukunft dazu beitragen, dass Protokolle effektiver und sicherer implementiert werden können
und somit die Sicherheit von Netzwerken und Systemen erhöht wird.