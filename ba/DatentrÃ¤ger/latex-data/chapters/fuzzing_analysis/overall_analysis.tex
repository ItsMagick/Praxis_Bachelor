%! Author = chaorn
%! Date = 01.09.24
\subsection{Vergleich der erhobenen Metriken}\label{subsec:vergleich-der-erhobenen-metriken}
In diesem Kapitel werden die erhobenen Metriken der Fuzzer \gls{afl}Net, Pulsar und boofuzz miteinander verglichen.
Die Metriken wurden anhand der Anzahl der generierten Eingaben pro Sekunde und der Anzahl der reproduzierten Bugs gemessen.\newline\newline
Bei der Reinen Betrachtung der Anzahl der gesendeten Eingaben pro Sekunde ist Pulsar der langsamste Fuzzer.
Aufgrund der Implementierung des Fuzzers ist es ohne selbst geschriebene Skripte nicht möglich, die von Pulsar generierten
Eingaben zu automatisiert an das zu testende Programm zu senden.
Die Anzahl der generierten Eingaben pro Sekunde, die an das \gls{zup} gesendet werden, ist somit vom Tester abhängig.
Die Frage der Automatisierung stellt sich hierbei als entscheidend heraus, da die manuelle Eingabe von Eingaben in das
\gls{zup} die Geschwindigkeit des Fuzzers erheblich beeinflusst.
Der dadurch einhergehende Mangel von Automatisierung führt dazu, dass ein weiteres Monitoring des \gls{zup} notwendig ist,
um die generierten Eingaben und den Zustand des \gls{zup} zu überwachen.\newline
Der nächst schnellere Fuzzer ist \gls{afl}Net mit durchschnittlich 8 Ausführungen pro Sekunde.
Bei der Ausführungsgeschwindigkeit von \gls{afl}Net ist zu beachten, dass der Fuzzer aufgrund der Fuzzing-Strategie
in Kombination des bereits integrierten Monitoring des \gls{zup} etwas langsamer ist.
Die Geschwindigkeit des Fuzzers ist jedoch ausreichend, um eine Vielzahl von Eingaben zu generieren und an das \gls{zup}
zu senden.
Außerdem anzumerken ist, dass \gls{afl}Net das \gls{zup} immer wieder erneut startet.
Das hat zur Folge, dass alle Zustände des \gls{zup} immer wieder zurückgesetzt werden und somit die Initialisierung des
Programms immer wieder erneut durchgeführt werden muss.
Eine Möglichkeit, um das Fuzzing-Verhalten zu beschleunigen, wäre einen Fuzzing-Harness für das Programm zu schreiben, welcher es
ermöglichen würde den \textit{persistent mode} von \gls{afl} zu verwenden.
Mit ihm ist es möglich einen Zustand des \gls{zup} statisch zu implementieren, welcher dem des Programms nach dem
Initialisieren des \gls{zup} entspricht.
Diese Technik ist jedoch sehr aufwändig und benötigt sehr umfangreiches Wissen über die Funktionsweise des \gls{zup},
welches nur durch Sourcecodeanalyse oder Reverse Engineering des bereits kompilierten Programms möglich ist.
Dieser Fuzzing-Harnes kann unter umständen dazu führen, dass wichtige Programmpfade und Kontrollflüsse des \gls{zup}
vernachlässigt oder weggelassen werden können und die Wahrscheinlichkeit von false positives erhöht.\newline
Der schnellste Fuzzer ist boofuzz mit durchschnittlich 77,88 Ausführungen pro Sekunde.
Diese Anzahl von Ausführungen pro Sekunde ist auf die Implementierung des Fuzzers zurückzuführen.
Boofuzz gehört zu der gattung der Batch-Mutation-Fuzzer und generiert eine Vielzahl von Eingaben auf einmal.
Mithlife dieser Technik ist es möglich, eine Vielzahl von Eingaben zu generieren, welche jedoch nach Analyse der gesammelten
Logs nach simplen Brute-Force angriffen aussehen.\newline\newline
Die Anzahl der Ausführungen pro Sekunde ist jedoch nicht ausschlaggebend für die Effektivität eines Fuzzers.
Sie wird anhand der Anzahl der gefundenen Bugs pro Zeit gemessen.
Die Anzahl der gefundenen Bugs beläuft sich bei \gls{afl}Net auf 3 Bugs, bei Pulsar und boofuzz auf keinen einzigen.
Trotz einer niedrigen Anzahl an Ausführungen pro Sekunde konnte \gls{afl}Net bereits in sieben Minuten (vgl.\ Abb.~\ref{fig:aflnet_bugs})
den ersten Bug finden.
Dies spricht für eine hohe Effektivität des Fuzzing-Ansatzes von code-coverage und der Fuzzing-Strategie von \gls{afl}Net.
Die Zunahme der Präzision von generierten validen Paketen spiegelt sich ebenso in der Grafik~\ref{fig:mqtt-aflnet_stdout}
wider.\newline\newline
Zusammenfassend ist in Hinsicht der Forschungsfrage \textit{Q1} (siehe~\ref{researc-questions}) festzustellen, dass
\gls{afl}Net im Rahmen der vollzogenen Experimente der effektivste Fuzzer ist.
Die Anzahl der gefundenen Bugs und die Effizienz der generierten Eingaben sprechen für eine hohe Effektivität des Fuzzers
im Gegensatz zu den anderen Fuzzern.
Im Hinblick auf die Ausführungsgeschwindigkeit ist boofuzz der schnellste Fuzzer, jedoch konnte er keine Bugs finden.
Stellt man beide Fuzzer einander gegenüber, so beläuft sich die Ausführungsgeschwindigkeit von \gls{afl}Net auf durchschnittlich
8 Ausführungen pro Sekunde (siehe Abb.~\ref{fig:aflnet_execs}), während boofuzz auf durchschnittlich 77,88 Ausführungen pro
Sekunde (siehe Abb.~\ref{fig:exec_speed_boo_normal}) kommt.\newline
Auch in der Hinsicht der Effizienz der generierten Eingaben ist \gls{afl}Net der effizienteste Fuzzer.
Die generierten Eingaben sind präzise und enthalten mit zunehmender Zeit der Kampagne weniger ungültige Eingaben (siehe Abb.~\ref{fig:mqtt-aflnet_stdout}).
Dies ist jedoch nicht der Fall bei boofuzz, welcher eine Vielzahl von Eingaben generiert, die nach Brute-Force-Angriffen
aussehen und direkt beim Empfangen von dem \gls{mqtt}-Broker verworfen werden~\ref{fig:boo_case_log}.\newline
Zu dem Fuzzer Pulsar konnten jedoch keine aufschlussreichen Ergebnisse erzielt werden.