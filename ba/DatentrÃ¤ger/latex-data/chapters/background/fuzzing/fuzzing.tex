%! Author = charon
%! Date = 2/1/24

\subsection{Fuzzing}\label{subsec:fuzzing}
Fuzzing ist eine Black Box Testmethode, die darauf abzielt, Fehler in Software zu finden, indem zufällige oder
semi-zufällige Daten als Eingabe verwendet werden~\cite{fuzzing}.
Diese Daten werden in der Regel von einem Fuzzer generiert, der die Software mit den Daten füttert und auf unerwartetes Verhalten prüft.
Fuzzing ist eine effektive Methode, um Fehler in Software zu finden, die durch unerwartete Eingaben verursacht werden.
Es ist eine weit verbreitete Methode, um Sicherheitslücken in Software zu finden, die von Hackern ausgenutzt werden können.
Fuzzing kann auch dazu verwendet werden, um die Stabilität und Zuverlässigkeit von Software zu testen und um sicherzustellen,
dass sie korrekt funktioniert. \\
Es gibt verschiedene Arten von Fuzzern, die unterschiedliche Strategien und Techniken verwenden, um Fehler in Software zu finden.
Die Implementierung der Techniken und Strategien hängen von den Anforderungen und Zielen des Fuzzers ab.
In den folgenden Abschnitten werden die verschiedenen Arten von Fuzzern, ihre Strategien und Techniken näher erläutert.
%! Author = charon
%! Date = 5/23/24

\subsubsection{Arten des Fuzzing}\label{subsubsec:arten-des-fuzzing}
Trotz dem, dass Fuzzing als eine Black-Box Testmethode gilt, gibt es verschiedene Arten von Fuzzing, die unterschiedliche
Strategien und Techniken verwenden, um Fehler in Software zu finden.
Die Arten des Fuzzing sind somit in drei Kategorien unterteilt~\cite{iot-fuzzing}:
\begin{itemize}
    \item Black-box Fuzzing
    \item White-box Fuzzing
    \item Grey-box Fuzzing
\end{itemize}
Jede Kategorie hat ihre eigenen Vor- und Nachteile und wird in den folgenden Abschnitten näher erläutert.\newline
Im Black-box Fuzzing wird die Software ohne Kenntnis des ursprünglichen Quellcodes getestet.
Der Fuzzer generiert zufällige oder semi-zufällige Eingaben und übergibt sie der Software.
Das Generieren der Eingaben erfolgt in der Regel durch Mutation von vorhandenen Eingaben oder durch Generierung neuer Eingaben
und wird durch Regeln limitiert.
Mit diesen Regeln wird sichergestellt, dass die Mehrheit der Eingaben nicht von der zu untersuchenden Software -- aufgrund
von bspw.\ Syntaxfehlern -- abgelehnt wird.
Bei der Durchführung des Fuzzing überwacht der Fuzzer das Verhalten der Software und prüft, ob unerwartetes Verhalten auftritt. \\
Das Paradigma, das hierbei verfolgt wird, ist das Testen der Software ohne Kenntnis des Quellcodes, des Kontrollflusses und somit der internen
Logik, um die Benutzung aus der Perspektive eines Nutzers der Software nachzustellen.\\
Die Vorteile dieser Herangehensweise sind, dass der Code nicht verfügbar sein muss und somit auch proprietäre Software getestet werden kann.
Zudem werden Fehler gefunden, welche durch einen Nutzer der Software ausgelöst werden können.
Der ausschlaggebende Nachteil dieser Vorgehensweise ist, dass die Eingaben aufgrund limitierter Informationen über die Software
nur erschwert zu komplexeren Bugs führen können.
Das hat zufolge, dass die gefundenen Bugs meistens einfacher Natur sind und Eingaben oftmals nicht in tiefe Verzweigungen des Codes
gelangen können~\cite{black-box-fuzzing}.\newline\newline
Im Gegensatz zum Black-box Fuzzing wird im White-box Fuzzing der Quellcode der Software benötigt.
Der Fuzzer generiert Eingaben, die auf der Analyse des Quellcodes basieren.
Dabei werden die Eingaben so generiert, dass sie die verschiedenen Pfade und Verzweigungen des Codes abdecken.
Das Ziel ist es, die Software mit Eingaben zu füttern, die potenzielle Fehler in den verschiedenen Teilen des Codes auslösen.
Durch die Kenntnis des Quellcodes kann der Fuzzer gezieltere Eingaben generieren und somit auch komplexere Fehler finden.
Der Vorteil dieser Methode ist, dass sie effektiver ist als Black-box Fuzzing, da sie gezieltere Eingaben generieren kann.
Sie kann auch dazu verwendet werden, um spezifische Teile des Codes zu testen und um sicherzustellen, dass sie korrekt funktionieren.
Der Nachteil dieser Methode ist, dass der Quellcode verfügbar sein muss, was bei proprietärer Software ein Problem darstellen kann.
Zudem kann das White-box Fuzzing aufgrund der Komplexität des Codes und der Anzahl der möglichen Pfade und Verzweigungen
sehr aufwendig sein~\cite{black-box-fuzzing}.\newline\newline
Grey-box Fuzzing ist eine Kombination aus Black-box und White-box Fuzzing.
Die Verwendung eines Gray-box Fuzzers erfordert keinen Zugriff auf Sourcecode, jedoch kann dieser optional verwendet werden.
Unter Verwendung des Quellcodes kann die Performance des Fuzzers verbessert werden, indem gezieltere Eingaben generiert werden.
Die Vorteile des Gray-box Fuzzing sind, dass es effektiver ist als Black-box Fuzzing, da es gezieltere Eingaben generieren kann.
Außerdem ermöglicht dieser Ansatz auch das Analysieren von proprietärer Software aufgrund dessen, dass der Quellcode nicht für das
Fuzzen von Software notwendig ist~\cite{grey-box-fuzzing}.
Die Herangehensweise des Grey-box Fuzzing ist ähnlich zu dem des Black-box Fuzzing.
Jedoch können vor dem Kompilieren des Quellcodes Instruktionen des Fuzzers eingefügt werden, um das Fuzzing zu verbessern.
Zudem wird unter bspw.\ \gls{afl} ein eigener Compiler verwendet, um Metadaten des Programms zu extrahieren und weitere
Instrumentierungsanweisungen in den Code zu injizieren~\cite{afl-instr}.
%! Author = charon
%! Date = 5/23/24
\subsubsection{Generierung von Input}\label{subsubsec:generierung-von-input}
Die Generierung von Input ist ein wichtiger Bestandteil des Fuzzing-Prozesses.
Hierbei unterscheiden sich die Fuzzer in ihrer Herangehensweise und den Techniken, die sie verwenden, um Eingaben zu generieren.
Die Generierung von Input kann auf verschiedene Arten erfolgen, die in den folgenden Abschnitten näher erläutert werden.\newline\newline
\textit{Generationsbasierende}~\cite{iot-fuzzing} Fuzzer generieren Eingaben, indem sie eine Menge von Eingaben in Epochen erstellen.
Jede Epoche wird als Generation bezeichnet.
Diese Eingaben basieren auf einem vom Anwender festgelegten Regelwerk.
Diese Herangehensweise des Generierens von Eingaben ist effektiv, um die Software mit Eingaben zu versorgen, welche nicht
bereits im Vorfeld von dem \gls{zup} als fehlerhaft befunden werden sollen.
Beispielsweise ist dies Vorteilhaft, wenn besonders komplexe Syntax des Input vonnöten ist.\newline\newline
\textit{Mutationsbasierende}~\cite{iot-fuzzing} Fuzzer generieren Eingaben, indem sie vorhandene Eingaben, welche bereits im Vorfeld definiert wurden, verändern.
Hierbei wird der Aufbau und die Struktur der Eingaben standardmäßig nicht berücksichtigt.
Diese Veränderung wird Mutation genannt und kann auf verschiedene Arten erfolgen.
Die Mutation kann beispielsweise durch das Hinzufügen, Entfernen oder Verändern von Bytes in der Eingabe erfolgen.
Um die Güte der neu generierten Eingaben zu bestimmen, wird die Code Coverage -- also die im Quellcode erreichte Tiefe --
analysiert.
Eingaben, die besonders viele Codesegmente erreichen werden als besonders gut gewertet und als bevorzugte Eingabe für
weitere Mutationen verwendet.

%! Author = charon
%! Date = 5/23/24

\subsubsection{Intelligenz von Fuzzern}\label{subsubsec:intelligenz-von-fuzzern}
Die Intelligenz verschiedener Fuzzer kann in zwei Kategorien~\cite{fuzzer-intelligence} unterteilt werden:
\begin{itemize}
    \item Brute-Force Fuzzer
    \item Intelligente Fuzzer
\end{itemize}
Zu den \textit{Brute-Force} Fuzzern gehören diejenigen, die das Feedback eines Programms nicht interpretieren.
Hierzu werden die Eingaben zufällig generiert und an das \gls{zup} übergeben.
Aufgrund dessen werden die Eingaben nicht auf ihre Gültigkeit oder Korrektheit überprüft.
Dumme Fuzzer sind in der Regel einfach zu implementieren und erfordern keine speziellen Kenntnisse über das \gls{zup}.
Sie sind jedoch weniger effektiv als intelligente Fuzzer~\cite{fuzzer-intelligence}, da sie keine Informationen über das Verhalten des Programms
sammeln und somit nicht in der Lage sind, gezieltere Eingaben zu generieren.
Ein Vorteil der Verwendung von dummen Fuzzern ist, dass sie in der Regel schneller sind als intelligente Fuzzer.
Das ist darauf zurückzuführen, dass aufgrund der nicht verwendeten Feedback-Schleife -- in der die Antworten des \gls{zup}
auf die Eingaben bewertet werden und anhand dessen eine passende Generierung der Eingaben gewählt wird --
die Eingaben schneller generiert werden können.\newline\newline
\noindent Intelligente Fuzzer adaptieren die Generierung der Eingaben anhand des Feedbacks.
Hierzu werden die Reaktion des Programms auf die Eingabe, die Anzahl der ausgeführten Instruktionen
oder die Anzahl der gefundenen Bugs analysiert.
Der Fokus der Analyse hängt von der Implementierung des Fuzzers ab.
Intelligente Fuzzer können das Verhalten des Programms überwachen und auf unerwartete Ereignisse reagieren~\cite{smart-fuzzing}.
Sie können auch Informationen über die Struktur des Programms sammeln und diese Informationen verwenden, um gezieltere Eingaben zu generieren.\newline
Intelligente Fuzzer sind in der Regel effektiver als dumme Fuzzer, da sie gezieltere Eingaben generieren können.
Sie sind jedoch auch komplexer zu implementieren und erfordern spezielle Kenntnisse über das \gls{zup}.

%! Author = charon
%! Date = 5/23/24
\subsubsection{Strategien von Fuzzern}\label{subsubsec:strategien-von-fuzzern}
Fuzzer können verschiedene Strategien verfolgen, um Eingaben zu generieren und das Verhalten der Software zu überwachen.
Die Strategien können in zwei Kategorien unterteilt~\cite{iot-fuzzing} werden:
\begin{itemize}
    \item Abdeckung von Codepfaden
    \item Zielgerichtetes Fuzzen
\end{itemize}
Bei dem zielgerichteten Fuzzen werden nur bestimmte Teile des Codes getestet.
Der Fuzzer generiert Eingaben, die auf bestimmten Kriterien basieren, um gezielt Fehler in diesen Teilen des Codes zu finden.
Zu den Kriterien gehören beispielsweise die Anzahl der ausgeführten Instruktionen, die Anzahl der gefundenen Bugs oder
die Reaktion des Programms auf die Eingabe.
Bei dieser Technik handelt es sich in der Regel um einen White-Box Fuzzer.
Um nur auf bestimmte Teile eines Programms zu fokussieren, wird der Quellcode benötigt~\cite{directed-greybox-fuzzing}.\newline\newline
Bei der Strategie der Codepfadabdeckung wird versucht, möglichst viele Pfade und Verzweigungen des Codes abzudecken.
Der Fuzzer generiert Eingaben, die verschiedene Pfade und Verzweigungen des Codes abdecken, um potenzielle Fehler in verschiedenen
Teilen des Codes zu finden.
Diese Technik wird in der Regel von Black- und Grey-Box Fuzzern verwendet, da sie keine Kenntnis des Quellcodes benötigen und sich
somit anderer Informationen bedienen müssen.
Die Codepfadabdeckung ist eine effektive Methode, um sicherzustellen, dass große Teile des Codes getestet wurden~\cite{iot-fuzzing}.
Somit ist es ebenso möglich besonders komplexe Bugs und Schwachstellen in einem System zu finden, da die Eingaben besonders
tief in den Programmcode gelangen.
