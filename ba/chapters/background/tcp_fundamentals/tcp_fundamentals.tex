%! Author = charon
%! Date = 6/20/24

\subsection{Grundlagen der TCP/IP-Kommunikation}\label{subsec:tcp/ip-kommunikation}
Die \gls{tcp}/\gls{ip}-Kommunikation ist ein Protokoll-Stack, der für die Kommunikation zwischen Computern verwendet wird.
Der Protokoll-Stack besteht aus zwei Schichten, dem \gls{tcp}- und dem \gls{ip}-Protokoll~\cite{tcp-ip}.
Das \gls{ip}-Protokoll ist für die Adressierung und das Routing von Datenpaketen zuständig.
Das \gls{tcp}-Protokoll ist für die zuverlässige Übertragung von Datenpaketen zuständig.
Die Kommunikation zwischen zwei Computern erfolgt über eine Verbindung, die durch eine \gls{ip}-Adresse und einen Port identifiziert wird.
Die \gls{ip}-Adresse identifiziert den Computer und der Port identifiziert den Dienst, der auf dem Computer ausgeführt wird.
Die Kommunikation erfolgt über eine Reihe von Datenpaketen, die zwischen den beiden Computern ausgetauscht werden.
Die Datenpakete enthalten Informationen über den Absender, den Empfänger, die Daten selbst und eine Prüfsumme,
die zur Überprüfung der Datenintegrität verwendet wird.
Die gesamte Kommunikation kann in drei Schritten: \textit{Verbindungsaufbau}, \textit{Datenübertragung} und \textit{Verbindungsabbau}
eingeteilt werden.
Der Verbindungsaufbau erfolgt in der Regel über eine Reihe von Schritten, die als \textit{Handshake} bezeichnet werden~\cite{tcp-handshake}.
Dieser Handshake besteht aus den drei Schritten \textit{SYN}, \textit{SYN-ACK} und \textit{ACK}.
Im Rahmen des Verbindungsaufbaus erfolgt die Verbindung sowie die Herstellung der Verbindung zwischen den beiden Computern.
Während der Datenübertragung erfolgt ein Austausch der Daten zwischen zwei Kommunikationspartnern.
Der Verbindungsabbau impliziert die Trennung der Verbindung.
Die Implementierung der \gls{tcp}/\gls{ip}-Kommunikation erfolgt in der Regel über die Socket-Programmierung.
Bei der Socket-Programmierung unter UNIX-Systemen hierzu die bereits im System implementierte Socket-API verwendet.
Die Socket-API stellt eine Reihe von Funktionen zur Verfügung, die es einem Programm ermöglichen, mit dem Netzwerk zu kommunizieren.
Sockets werden in der Regel als Datei-Handle dargestellt und können vom Kernel gelesen und geschrieben werden.
Diese Dateien sind in der Regel nicht im Namespace des Dateisystems abgelegt.
Metadaten der Sockets werden jedoch im Dateisystem abgelegt und können zu Analysezwecken von weiteren Programmen verwendet werden.
Diese Metadaten können im Dateisystem unter \textit{/proc/net/tcp}, \textit{/proc/net/tcp6} und unter im Verzeichnis
\textit{/proc/sys/net} eingesehen werden~\cite{tcp-manpage}.\newline
Der Gesamte Stack kann in vier Schichten unterteilt werden:
\begin{itemize}
    \item Anwendungsschicht
    \item Transportschicht
    \item Internetschicht
    \item Netzwerk-Interface-Schicht
\end{itemize}
Die Anwendungsschicht ist die oberste Schicht des Protokoll-Stacks und enthält die Anwendungen,
die die Kommunikation zwischen den Computern ermöglichen.
Zu dieser Schicht gehören die von einer Anwendung verwendeten Protokolle, wie z.B.\ \gls{http}, \gls{ftp}, \gls{dns} oder \gls{smtp},
mithilfe derer Daten über ein Netzwerk übertragen werden.
Die Funktion der zweiten Schicht des Protokoll-Stacks, der sogenannten Transportschicht, besteht in der Übertragung von Datenpaketen.
In dieser Schicht obliegt die zuverlässige Übertragung von Datenpaketen dem \gls{tcp}-Protokoll.
Die Internetschicht stellt die dritte Schicht des Protokoll-Stacks dar und ist für die Adressierung sowie das Routing von Datenpaketen zuständig.
In dieser Schicht obliegt dem \gls{ip}-Protokoll die Adressierung und das Routing von Datenpaketen.
Die Netzwerkschicht stellt die unterste Schicht des Protokoll-Stacks dar und ist für die Verbindung mit einem Netzwerk
sowie der dazugehörigen Hardware zuständig.\newline
In dieser Arbeit wird die Anwendungsschicht des Protokoll-Stacks analysiert.