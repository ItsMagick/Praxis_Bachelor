%! Author = charon
%! Date = 5/23/24

\subsubsection{Intelligenz von Fuzzern}\label{subsubsec:intelligenz-von-fuzzern}
Die Intelligenz verschiedener Fuzzer kann in zwei Kategorien~\cite{fuzzer-intelligence} unterteilt werden:
\begin{itemize}
    \item Brute-Force Fuzzer
    \item Intelligente Fuzzer
\end{itemize}
Zu den \textit{Brute-Force} Fuzzern gehören diejenigen, die das Feedback eines Programms nicht interpretieren.
Hierzu werden die Eingaben zufällig generiert und an das \gls{zup} übergeben.
Aufgrund dessen werden die Eingaben nicht auf ihre Gültigkeit oder Korrektheit überprüft.
Dumme Fuzzer sind in der Regel einfach zu implementieren und erfordern keine speziellen Kenntnisse über das \gls{zup}.
Sie sind jedoch weniger effektiv als intelligente Fuzzer~\cite{fuzzer-intelligence}, da sie keine Informationen über das Verhalten des Programms
sammeln und somit nicht in der Lage sind, gezieltere Eingaben zu generieren.
Ein Vorteil der Verwendung von dummen Fuzzern ist, dass sie in der Regel schneller sind als intelligente Fuzzer.
Das ist darauf zurückzuführen, dass aufgrund der nicht verwendeten Feedback-Schleife -- in der die Antworten des \gls{zup}
auf die Eingaben bewertet werden und anhand dessen eine passende Generierung der Eingaben gewählt wird --
die Eingaben schneller generiert werden können.\newline\newline
\noindent Intelligente Fuzzer adaptieren die Generierung der Eingaben anhand des Feedbacks.
Hierzu werden die Reaktion des Programms auf die Eingabe, die Anzahl der ausgeführten Instruktionen
oder die Anzahl der gefundenen Bugs analysiert.
Der Fokus der Analyse hängt von der Implementierung des Fuzzers ab.
Intelligente Fuzzer können das Verhalten des Programms überwachen und auf unerwartete Ereignisse reagieren~\cite{smart-fuzzing}.
Sie können auch Informationen über die Struktur des Programms sammeln und diese Informationen verwenden, um gezieltere Eingaben zu generieren.\newline
Intelligente Fuzzer sind in der Regel effektiver als dumme Fuzzer, da sie gezieltere Eingaben generieren können.
Sie sind jedoch auch komplexer zu implementieren und erfordern spezielle Kenntnisse über das \gls{zup}.
