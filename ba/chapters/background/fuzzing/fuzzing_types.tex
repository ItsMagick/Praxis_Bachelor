%! Author = charon
%! Date = 5/23/24

\subsubsection{Arten des Fuzzing}\label{subsubsec:arten-des-fuzzing}
Trotz dem, dass Fuzzing als eine Black-Box Testmethode gilt, gibt es verschiedene Arten von Fuzzing, die unterschiedliche
Strategien und Techniken verwenden, um Fehler in Software zu finden.
Die Arten des Fuzzing sind somit in drei Kategorien unterteilt~\cite{iot-fuzzing}:
\begin{itemize}
    \item Black-box Fuzzing
    \item White-box Fuzzing
    \item Grey-box Fuzzing
\end{itemize}
Jede Kategorie hat ihre eigenen Vor- und Nachteile und wird in den folgenden Abschnitten näher erläutert.\newline
Im Black-box Fuzzing wird die Software ohne Kenntnis des ursprünglichen Quellcodes getestet.
Der Fuzzer generiert zufällige oder semi-zufällige Eingaben und übergibt sie der Software.
Das Generieren der Eingaben erfolgt in der Regel durch Mutation von vorhandenen Eingaben oder durch Generierung neuer Eingaben
und wird durch Regeln limitiert.
Mit diesen Regeln wird sichergestellt, dass die Mehrheit der Eingaben nicht von der zu untersuchenden Software -- aufgrund
von bspw.\ Syntaxfehlern -- abgelehnt wird.
Bei der Durchführung des Fuzzing überwacht der Fuzzer das Verhalten der Software und prüft, ob unerwartetes Verhalten auftritt. \\
Das Paradigma, das hierbei verfolgt wird, ist das Testen der Software ohne Kenntnis des Quellcodes, des Kontrollflusses und somit der internen
Logik, um die Benutzung aus der Perspektive eines Nutzers der Software nachzustellen.\\
Die Vorteile dieser Herangehensweise sind, dass der Code nicht verfügbar sein muss und somit auch proprietäre Software getestet werden kann.
Zudem werden Fehler gefunden, welche durch einen Nutzer der Software ausgelöst werden können.
Der ausschlaggebende Nachteil dieser Vorgehensweise ist, dass die Eingaben aufgrund limitierter Informationen über die Software
nur erschwert zu komplexeren Bugs führen können.
Das hat zufolge, dass die gefundenen Bugs meistens einfacher Natur sind und Eingaben oftmals nicht in tiefe Verzweigungen des Codes
gelangen können~\cite{black-box-fuzzing}.\newline\newline
Im Gegensatz zum Black-box Fuzzing wird im White-box Fuzzing der Quellcode der Software benötigt.
Der Fuzzer generiert Eingaben, die auf der Analyse des Quellcodes basieren.
Dabei werden die Eingaben so generiert, dass sie die verschiedenen Pfade und Verzweigungen des Codes abdecken.
Das Ziel ist es, die Software mit Eingaben zu füttern, die potenzielle Fehler in den verschiedenen Teilen des Codes auslösen.
Durch die Kenntnis des Quellcodes kann der Fuzzer gezieltere Eingaben generieren und somit auch komplexere Fehler finden.
Der Vorteil dieser Methode ist, dass sie effektiver ist als Black-box Fuzzing, da sie gezieltere Eingaben generieren kann.
Sie kann auch dazu verwendet werden, um spezifische Teile des Codes zu testen und um sicherzustellen, dass sie korrekt funktionieren.
Der Nachteil dieser Methode ist, dass der Quellcode verfügbar sein muss, was bei proprietärer Software ein Problem darstellen kann.
Zudem kann das White-box Fuzzing aufgrund der Komplexität des Codes und der Anzahl der möglichen Pfade und Verzweigungen
sehr aufwendig sein~\cite{black-box-fuzzing}.\newline\newline
Grey-box Fuzzing ist eine Kombination aus Black-box und White-box Fuzzing.
Die Verwendung eines Gray-box Fuzzers erfordert keinen Zugriff auf Sourcecode, jedoch kann dieser optional verwendet werden.
Unter Verwendung des Quellcodes kann die Performance des Fuzzers verbessert werden, indem gezieltere Eingaben generiert werden.
Die Vorteile des Gray-box Fuzzing sind, dass es effektiver ist als Black-box Fuzzing, da es gezieltere Eingaben generieren kann.
Außerdem ermöglicht dieser Ansatz auch das Analysieren von proprietärer Software aufgrund dessen, dass der Quellcode nicht für das
Fuzzen von Software notwendig ist~\cite{grey-box-fuzzing}.
Die Herangehensweise des Grey-box Fuzzing ist ähnlich zu dem des Black-box Fuzzing.
Jedoch können vor dem Kompilieren des Quellcodes Instruktionen des Fuzzers eingefügt werden, um das Fuzzing zu verbessern.
Zudem wird unter bspw.\ \gls{afl} ein eigener Compiler verwendet, um Metadaten des Programms zu extrahieren und weitere
Instrumentierungsanweisungen in den Code zu injizieren~\cite{afl-instr}.