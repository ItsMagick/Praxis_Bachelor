%! Author = charon
%! Date = 5/23/24
\subsubsection{Strategien von Fuzzern}\label{subsubsec:strategien-von-fuzzern}
Fuzzer können verschiedene Strategien verfolgen, um Eingaben zu generieren und das Verhalten der Software zu überwachen.
Die Strategien können in zwei Kategorien unterteilt~\cite{iot-fuzzing} werden:
\begin{itemize}
    \item Abdeckung von Codepfaden
    \item Zielgerichtetes Fuzzen
\end{itemize}
Bei dem zielgerichteten Fuzzen werden nur bestimmte Teile des Codes getestet.
Der Fuzzer generiert Eingaben, die auf bestimmten Kriterien basieren, um gezielt Fehler in diesen Teilen des Codes zu finden.
Zu den Kriterien gehören beispielsweise die Anzahl der ausgeführten Instruktionen, die Anzahl der gefundenen Bugs oder
die Reaktion des Programms auf die Eingabe.
Bei dieser Technik handelt es sich in der Regel um einen White-Box Fuzzer.
Um nur auf bestimmte Teile eines Programms zu fokussieren, wird der Quellcode benötigt~\cite{directed-greybox-fuzzing}.\newline\newline
Bei der Strategie der Codepfadabdeckung wird versucht, möglichst viele Pfade und Verzweigungen des Codes abzudecken.
Der Fuzzer generiert Eingaben, die verschiedene Pfade und Verzweigungen des Codes abdecken, um potenzielle Fehler in verschiedenen
Teilen des Codes zu finden.
Diese Technik wird in der Regel von Black- und Grey-Box Fuzzern verwendet, da sie keine Kenntnis des Quellcodes benötigen und sich
somit anderer Informationen bedienen müssen.
Die Codepfadabdeckung ist eine effektive Methode, um sicherzustellen, dass große Teile des Codes getestet wurden~\cite{iot-fuzzing}.
Somit ist es ebenso möglich besonders komplexe Bugs und Schwachstellen in einem System zu finden, da die Eingaben besonders
tief in den Programmcode gelangen.