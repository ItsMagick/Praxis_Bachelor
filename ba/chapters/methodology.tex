%! Author = chaorn
%! Date = 08.09.24

\section{Methodik}\label{sec:methodology}
In diesem Kapitel wird die Methodik zur Analyse der Performance von Netzwerkfuzzern vorgestellt.
Die Methodik umfasst die Auswahl der Fuzzer, die Definition der Testumgebung, die Auswahl der Testfälle und die
Durchführung der Tests.\newline\newline
Um möglichst viele potenzielle Schwachstellen, die das \gls{zup} aufweisen könnte, zu identifizieren, wurde das
\gls{zup} mit einem Address Sanitizer am Vorbild von Holz~\cite{mutuation-analysis} kompiliert.
Der Address Sanitizer ist ein Werkzeug, das von der llvm-Toolchain bereitgestellt wird und zur Identifizierung von
Speicherfehlern in C- und C++-Programmen verwendet wird.
Der Address Sanitizer überwacht den Speicherzugriff und identifiziert potenzielle Speicherfehler wie Buffer Overflows,
Memory Leaks, Use-after-free, Use-after-return, Use-after-scope und Double-free Bugs erkennt und das Programm kontrolliert
mit einem Stoppsignal zum Absturz führt~\cite{asan}.\newline
Die Tests wurden auf einem Rechner mit einem Intel Core i9-9900K Prozessor und 64 GB RAM durchgeführt.
Das Betriebssystem des Rechners ist zur Zeit der Tests Parrot Security 6.2 (lorikeet), welches auf Debian basiert.
Die Tests wurden bis auf Tests mit Pulsar auf der Host Maschine durchgeführt.
Für Pulsar wurde eine virtuelle Umgebung mit Docker~\cite{pulsar-docker} eingerichtet.\newline\newline
Alle Kampagnen wurden mit einer Laufzeit von 48 Stunden durchgeführt.
Zur Analyse der Performance der Fuzzer werden verschiedene Metriken verwendet.
Die Metriken umfassen die Anzahl der gefundenen potenziellen Schwachstellen, die Ausführungsgeschwindigkeit und die
Effizienz der generierten Testfälle anhand der erfolgreichen Verbindungen mit dem \gls{mqtt}-Broker.
Für das Loggen der Metriken wurde ein Skript entwickelt, das die Ausgabe des \gls{zup} von stdout und stderr in eine Datei
schreibt.
Anhand dieser Dateien können die aufgetretenen Fehler mit deren Stacktrace des Address Sanitizers und die Anzahl der erfolgreichen
Verbindungen mit dem \gls{mqtt}-Broker ermittelt werden.
Da das jedoch nicht unter \gls{afl}Net möglich ist, wurde hierzu eine preloaded library entwickelt, die die Ausgabe des
Address Sanitizers und des stdin-Streams in eine Datei umleitet.
Die Metriken werden in Diagrammen visualisiert und miteinander verglichen, um die Performance der Fuzzer zu bewerten.
Das Ziel der Analyse ist es, die Effektivität der Fuzzer bei der Identifizierung von Schwachstellen in Netzwerkprotokollen
zu bewerten und die Unterschiede in der Performance der Fuzzer zu identifizieren.
Die Ergebnisse der Tests werden in Kapitel~\ref{subsec:vergleich-der-erhobenen-metriken} vorgestellt und diskutiert.
Hierbei wird die Forschungsfrage \textit{Q1}~\ref{researc-questions} aus Kapitel~\ref{sec:einleitung} beantwortet.