%! Author = chaorn
%! Date = 12.03.24

\clearpage
\newgeometry{top=2.4cm, bottom=2.4cm}
\pagestyle{empty}
\section*{Abstract}
Die Arbeit \emph{\mytitle} beschäftigt sich mit dem Fuzzing eines Netzwerkprotokolls und der damit verbundenen
Analyse eines bereits kompilierten Programms.
Nach der Einordnung und Kategorisierung des zu untersuchenden Programms werden grundlegende Begriffe und Herangehensweisen
der Programmanalyse und des Fuzzing mit \emph{AFL} erarbeitet. \\
\linebreak
Folgend wird eine detaillierte Übersicht über den in dieser Arbeit verwendeten Fuzzer \emph{AFL++} gegeben.
Hierzu werden die verwendete Strategie und die verbauten Technologien des Fuzzers beschrieben.
Anand der Funktionsweise des Fuzzers werden die mit ihm verbundenen Risiken offengelegt. \\
\linebreak
Im folgenden praktischen Teil der Arbeit werden die entwickelten und verwendeten Testumgebungen erklärt.
Dazu werden die Unterschiede der beiden entwickelten Ansätze aufgezeigt und ein Fazit zur Benutzbarkeit der
Testumgebungen gezogen.\\
\linebreak
Im Anschluss wird die Vorgehensweise zum Fuzzen eines Netzwerkprotokolls auf Applikationsebene erläutert.
Dazu gehört das Implementieren einer Preload-Library zum Überschreiben von Syscalls, die für die Kommunikation
über das Netzwerk verantwortlich sind.
Es werden drei Ansätze zur Leistungssteigerung beschrieben, indem die
von AFL++ bereitgestellten Funktionen zur Minimierung der Eingaben, zur Verwendung des Persistent Modes und zur Skalierung
der Kampagne über mehrere CPU-Kerne genutzt werden.\\
\linebreak
Abschließend werden die vom Fuzzer erzeugten Ausgaben der Statusanzeige des laufenden Fuzzers erklärt.
Die erzeugten Dateien und deren Inhalt werden anschließend ausgewertet und der Erfolg des Fuzzingprozesses ermittelt. \\
\linebreak
Zum Abschluss werden die während der Arbeit aufgetretenen Probleme offengelegt und die Arbeit reflektiert.
Der verwendete Fuzzer kann als Erweiterung dieser Arbeit durch die Bibliothek \emph{LibAFL} erweitert und den Anforderungen
der Arbeit angepasst werden.

\restoregeometry