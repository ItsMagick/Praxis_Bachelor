\clearpage
\section{Ausblick}\label{sec:ausblick}
Aufgrund der beliebig steigenden Komplexität der Herangehensweisen beim fuzzen gibt es bereits erste Baukastensysteme zum Bauen
eigener Fuzzer.
Von \gls{aflpp} selbst gibt es eine in RUST geschriebene Bibliothek LibAFL~\cite{libafl} zum Bauen modularer Fuzzer.
Diese Art des Fuzzen erlaubt es dem Tester, noch schnellere Ausführgeschwindigkeiten zu erzielen.
Mithilfe dieser Frameworks ist es möglich, Fuzzing komplett Plattformunabhängig durchzuführen.
Zudem ist es möglich, den mit LibAFL gebauten fuzzer in ein gewünschtes Binary zu integrieren und somit natives Fuzzing
zu ermöglichen.\\
Nachdem sich in dieser Arbeit mit \gls{aflpp} auseinandergesetzt wurde, soll sich in der auf dieser Arbeit aufbauenden
Bachelorarbeit mit den alternativen Ansätzen des Fuzzing beschäftigt werden.
Dazu gehören die Abwägungen von Netzwerk basierenden Fuzzern wie boofuzz und modernen Baukastensystemen wie der bereits genannten
Fuzzing-Bibliothek LibAFL\@. \\
\linebreak
In einer zukünftigen Arbeit soll ein -- an das untersuchte Binary -- angepasster Fuzzer mithilfe des bereits angesprochenen
Frameworks LibAFL umgesetzt werden.
Hierbei soll die Performance der bereits existierenden Kampagne weiter optimiert werden.