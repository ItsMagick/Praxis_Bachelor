%! Author = charon
%! Date = 2/7/24
\section{Probleme}\label{sec:probleme}
Im Verlauf der Bearbeitung der Aufgabenstellung traten Probleme auf, deren Lösung im Folgenden aufgezeigt wird. \\
\linebreak
Das erste Problem, welches sich bereits am Anfang der Arbeit stellte, war die Erstellung eines Docker Containers.
Dieser Docker Container sollte es ermöglichen, die in dieser Arbeit verwendeten shared library mithilfe einer alten,
vom Binary genutzten \gls{gcc}-Version, für \gls{arm} zu kompilieren. \\
Der ursprüngliche Ansatz der Aufgabe bestand daraus, die Instruktionen aus dem Dockerfile eines bereits vorhandenen \gls{arm}-Containers
aus Dockerhub~\cite{docker-gcc} herauszukopieren und diese auf die gestellte Anforderung der \gls{gcc}-Version 4.8.1 anzupassen.
Dieser Ansatz stellte sich jedoch schnell als sehr Zeitintensiv heraus, da das Nachvollziehen fremder Logik innerhalb eines
Dockerfiles viel Zeit in Anspruch nimmt.
Der nächste Ansatz war es, ein Installationsskript anhand der Dokumentation von \gls{gcc}~\cite{gcc-doc} selbst zu erstellen.
Dieser Ansatz war jedoch erfolglos.
Die Schwierigkeit bestand daraus, die von der Version des Compilers benötigten Dependencies zu erfüllen.
Eine effiziente Lösung des Problems war es, ein altes Image zu verwenden, welches ein Betriebssystem aus dem Erscheinungsjahr
der \gls{gcc}-Version verwendet.
Daraufhin konnte die Installation des Compilers begonnen werden.
Da das Kompilieren von \gls{gcc} sehr Zeit- und Ressourcenintensiv (ca.\ 20 Minuten) ist, wurden mehrere Instanzen des Docker Containers gestartet,
um parallel an anderen Lösungsansätzen beim Auftreten eines Fehlers zu arbeiten.
Diese Fehler waren jedoch der Inkompatibilität der auf dem Image verwendeten glibc Version geschuldet.
Aufgrund der Komplexität und des hohen Zeitaufwands dieser Aufgabe wurde der Ansatz des Bauens eines eigenen Docker Image
verworfen.
Der letzte und schlussendlich eingesetzte Ansatz ist das Suchen eines bereits funktionierenden \gls{gcc} Binaries, das
für eine auf Debian basierte Linux Distribution gebaut wurde.
Dafür wurde in den Archiven der Debian und Ubuntu Repositories~\cite{ubuntu-archive} nach einer kompatiblen \gls{gcc}-Version gesucht.
Die kompatible \gls{gcc}-Version wurde dann innerhalb eines Shellscripts in einem für \gls{arm} gebauten Container
installiert und konnte dann verwendet werden.\\
\linebreak
Bei der Entwicklung des Testsystems kam es ebenso zu einigen schwerwiegenden Problemen.
Eines der Probleme beinhaltet die Entwicklung der chroot Umgebung.
Bei dem Mounten der verschiedenen Verzeichnisbäume in eine isolierte Umgebung kann es zu unerwünschten Beierscheinungen,
wie dem Neudefinieren von für das Betriebssystem notwendigen Verzeichnissen.
Ebenfalls hat das chroot Skript eine Bereinigungsfunktion, welche den Zustand des im Gitlab~\cite{gitlab} enthaltenen rootfs berücksichtig.
Diese Funktion löscht alle neu angelegten Dateien, welche nicht im Git Repository nachverfolgt werden, um möglichst nah am
Ausgangszustand der Firmware zu bleiben.
Somit kann es passieren, dass falsch gemountete Verzeichnisse gelöscht werden und damit für das Betriebssystem wichtige Dateien
verloren gehen.
Es ist also ratsam, eine virtuelle Maschine zu Testzwecken des chroot Skripts zu verwenden, damit das eigene Betriebssystem
unversehrt bleibt.