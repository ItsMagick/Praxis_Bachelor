%! Author = charon
%! Date = 2/8/24
\subsection{Internet of Things Gerätesicherheit}\label{subsec:internet-of-things-als-trusted-network}
~\gls{iot} kann wie folgt definiert werden: \\
\textit{"Das Internet of Things ist eine Gruppe von Infrastrukturen, welche verbundene Objekte miteinander verbindet und somit das
Verwalten, Data-Mining und die Verfügbarkeit der von ihnen generierten Daten stellen."}\cite[vgl.][2]{iot-definition} \\
\linebreak
Das zu untersuchende Gerät erfüllt die Charakteristik der Interkonnektivität, Stellung von Daten und der Sensorik.
Umgesetzt werden diese Merkmale mittels eines \gls{tcp}/~\gls{ip}-Stack, mit dem dieses Gerät gesteuert werden kann.
Oftmals wird die hohe Verfügbarkeit und Bedienbarkeit zur erleichterten Bedienung und Automatisierung des Alltags gewünscht.
Diese Funktionalitäten stehen jedoch oftmals im Kontrast zur Gerätesicherheit von Infrastrukturen und
der Integrität der darin enthaltenen Daten.\\
\linebreak
\gls{iot}-Geräte sind im Bestfall in einem separaten Netzwerk mit einer sehr restriktiven Konfiguration.
Von diesem Szenario ist jedoch nicht immer auszugehen, da gerade kleine und mittelständische Unternehmen und
Institutionen nicht immer das Know-how zu einer sauberen Trennung von Netzwerken mit sich bringen.
In solchen Szenarien können gerade Geräte mit Protokollen, mit denen es möglich ist, Geräte fernzusteuern, verheerende Folgen
auf die Gesundheit eines Netzwerkes haben.
Dieser Protokolle kann sich ein erfahrener Angreifer zu eigen machen, um unbefugten Zugang zur Infrastruktur eines Netzwerks
zu erlangen.\\
\linebreak
Um die Gerätesicherheit solcher Geräte weiter zu stärken und den Einfallswinkel potenzieller Schwachstellen möglichst
gering zu halten, wird der \gls{tcp}-\gls{ip}-Stack genauer analysiert.
