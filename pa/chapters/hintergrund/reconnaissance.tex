%! Author = charon
%! Date = 2/8/24

\subsection{Reconnaissance}\label{subsec:reconnaissance}
Um ein genaueres Bild des \gls{iot}-Geräts zu erhalten, wurden Informationen über das System gesammelt.
Dieser Prozess wird in der Cyber Security auch als \textit{Reconnaissance} bezeichnet.
Reconnaissance kann in die zwei Bereiche \textit{aktive} und \textit{passive} Reconnaissance eingeteilt werden.
Aktive Reconnaissance beschäftigt sich dabei hauptsächlich mit der direkten Interaktion mit einem System.
Bei der passiven Reconnaissance hingegen beschäftigt man sich mit der Recherche ohne Interaktion mit dem System.
Dieses Kapitel macht Gebrauch von beiden Aspekten dieses Konzepts.\\
\linebreak
Die im Kapitel~\ref{sec:einleitung} beschriebene Path-Traversal-Schwachstelle resultierte darin, dass ein Snapshot des ~\gls{iot}-
Geräts und somit auch der Firmware gemacht werden konnte.
Die Firmware beinhaltet ein Hauptbinary mit dem Namen \textit{mmapp}, das für alle Benutzerfunktionen des Geräts
verantwortlich ist.
Bei diesem Binary handelt es sich um ein dynamisch gelinktes Closed-Source-\gls{elf}-Binary in einer 32-Bit-Architektur.
Die Plattform, für die das Binary kompiliert wurde, ist eine ~\gls{arm}v7 Plattform.
%! Author = chaorn
%! Date = 10.03.24

\begin{lstlisting}[language=bash, caption={Zeigt die Ausgabe des \texttt{file} Befehls auf das Hauptbinary mmapp.
\texttt{file} ist ein Programm, welches versucht eine gegebende Datei zu klassifizieren.},label={lst:file-mmapp}]
$ file mmapp
mmapp: ELF 32-bit LSB executable, ARM, EABI5 version 1 (SYSV), dynamically linked, interpreter /lib/ld-linux-armhf.so.3, for GNU/Linux 2.6.16, BuildID[sha1]=618c3b2ab4868d6d25480a6c1c9d88498c4d7379, stripped
\end{lstlisting}
Da das Binary dynamisch gelinkt ist, ist davon auszugehen, dass es von Bibliotheken außerhalb des Binarys abhängig ist.
Während der Laufzeit des Programms werden die benötigten Bibliotheken in den Adressraum des Programms geladen.
In Listing~\ref{lst:ldd-mmapp} sind die Bibliotheken aufgeführt, die von mmapp verwendet werden.
Ein Portscan mit~\gls{nmap} zeigt offene~\gls{tcp}-Ports.
\pagebreak
%! Author = chaorn
%! Date = 10.03.24
\lstdefinestyle{nmap}{
    columns=fixed,
}

\begin{lstlisting}[style=nmap, language=bash, caption={Zeigt die Ausgabe des Portscans des Geräts mit nmap. Die hier verwendeten Flags \texttt{-sV -p-}
                            bezwecken die Ausgabe der Version der Services aller Ports.},label={lst:nmap-scan}]
$ sudo nmap -sV -p- 10.0.0.10
Starting Nmap 7.93 (https://nmap.org) at 2023-10-31 14:31 CET
Nmap scan report for 10.0.0.10
Host is up (0.00030s latency).
Not shown: 65529 closed tcp ports (reset)
PORT       STATE  SERVICE      VERSION
22/tcp     open   ssh          Dropbear sshd 2022.83
80/tcp     open   http         Node.js Express framework
3060/tcp   open   http         Node.js
4352/tcp   open   pjlink       PJLink projector control
7142/tcp   open   unknown
41794/tcp  open   tcpwrapped
\end{lstlisting}
Die Ports \textit{80} und \textit{3060} verweisen auf ein öffentlich zugängliches Webinterface.
Zusätzlich befindet sich auf dem Port \textit{4352} ein textbasiertes Protokoll namens \textit{pjlink}.
Der für diese Arbeit relevante Port ist heit der Port mit der Nummer \textit{7142}.
Laut der Dokumentation des Herstellers ~\cite{binary-prot-doc}, existiert ein
Netzwerkprotokoll zur Fernsteuerung des Geräts auf dem offenen ~\gls{tcp}-Port 7142.
Dieses Protokoll erwartet Bytes als Input.
Für die Kommunikation mit dem Gerät über~\gls{tcp} wird ein Python-Skript~\cite{binaryProt-py} verwendet, das zunächst nur die in der Dokumentation festgehaltenen Befehle enthält.
Die nach dem Ausführen des Scripts erhaltenen Antworten werden in einer Tabelle~\cite{reqs-n-res} festgehalten. \\
Das Senden der Befehle erfordern das Mitgeben einer Checksumme.
Die Checksumme berechnet sich aus der Summer aller im Befehl enthaltenen Bytes.
Im Anschluss wird nur das letzte Byte betrachtet, das der mitzusendenden Checksumme entspricht.