\section{Fazit}\label{sec:fazit}
In dieser Praxisarbeit zum Thema des Fuzzings eines Netzwerkprotokolls eines~\gls{iot} Binarys wurde die grundlegende
Herangehensweise zum Fuzzen und zum Entwickeln einer verwendbaren Testumgebung vermittelt.
Quereinsteiger in das Thema Fuzzing haben so die Möglichkeit, möglichst schnell zu einem erwünschten Ziel der Entwicklung einer
lauffähigen Fuzzingumgebung zu kommen.
Zudem wurden Erkenntnisse in die Funktionsweise von~\gls{afl} dargestellt, welche die Einarbeitung in das Thema Fuzzen
mit~\gls{afl} vereinfachen sollte.\\
Des Weiteren wurden Methodiken zum Optimieren der Fuzzing Kampagne, wie desocketing einer Netzwerk basierenden Applikation beleuchtet,
welche es neuen Nutzern von~\gls{afl} ermöglicht, möglichst performante Kampagnen zu starten.
Das sollte die Wahrscheinlichkeit der Findung neuer Bugs und Schwachstellen erhöhen. \\
Als Nächstes wurden die Möglichkeiten zur Auswertung einer erfolgreichen Kampagne erläutert.
Diese ermöglichen es dem Nutzer des~\gls{afl}, die Ausgaben der Kampagne zu deuten und potenzielle Schwachstellen
mithilfe gezielter Instrumentation zu verifizieren.\\
Als Letztes wurden aufgetretene Probleme angesprochen und deren Lösungswege offengelegt.
Das Kapitel Probleme soll es der Leserschaft der Arbeit ermöglichen bei ähnlichen Problemen möglichst schnell
zu einem Lösungsweg zu kommen und als Hilfestellung dienen.