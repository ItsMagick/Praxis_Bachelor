%! Author = charon
%! Date = 2/8/24

\subsection{Aufbau der Testumgebung mit chroot}\label{subsec: chroot}
Das Programm chroot ermöglicht das Ändern des aktuellen Wurzelverzeichnisses innerhalb eines bereits laufenden Betriebssystems.
Unter vielen Linux-Distributionen ist das Wurzelverzeichnis des Dateisystems unter dem Verzeichnispfad \textbf{/} zu finden.
Die minimale Struktur eines Linux \gls{fs}~\cite{root-fs} beinhaltet die Verzeichnisse \textit{/boot, /dev, /etc, /bin, /sbin}
und in einigen Fällen \textit{/tmp}.
Hierbei befinden sich unter \textit{/boot} Informationen zum Starten des Betriebssystems, unter \textit{/dev} Informationen zur verwendeten Hardware,
unter \textit{/etc} Konfigurationsdateien für das Betriebssystem und unter \textit{/bin} und \textit{/sbin} ausführbare Dateien (Executables). \\
Im Spezialfall des zu untersuchenden Binarys werden zum Start weitere Verzeichnisse benötigt.
Dazu gehören die Verzeichnisse \textit{/proc}~\cite{arch-procfs}, mit den darin enthaltenen Informationen zum laufenden System und Kernel,
\textit{/sys}~\cite{kernel-sysfs} mit enthaltenen Datenstrukturen, welche vom Betriebssystem verwendet werden
und \textit{/run}~\cite{runfs}, welches Systeminformationen nach dem Bootprozess des Betriebssystems enthält.
%! Author = chaorn
%! Date = 21.02.24

\begin{lstlisting}[language=bash, caption={Mounten der von chroot benötigten Verzeichnisse},label={lst:chroot-mount}]
$ sudo mount -t proc root/proc root/proc/
$ sudo mount -t sysfs root/sys root/sys/
$ sudo mount --rbind /dev root/dev/
$ sudo mount --rbind root/run root/run/
$ sudo mount --rbind /lib64 root/lib64/
$ sudo mount --rbind root/tmp root/tmp/
\end{lstlisting}
Das Mounten von \textit{/lib64} dient in diesem Script~\cite{chroot-fuzz} nur als Komfort, um nicht alle Abhängigkeiten für
\gls{afl} manuell kopieren zu müssen.
Eine Alternative dazu wäre, das \gls{afl}-Executable \texttt{afl-fuzz} statisch zu linken, damit es ohne Abhängigkeiten
ausführbar ist.
Diese Möglichkeit ist bereits im Quellcode von \gls{aflpp} eingebaut.\\
Bevor das Programm gefuzzt werden kann, muss die Umgebung mit entsprechenden Umgebungsvariablen angepasst werden.
Hierbei ist die \textit{PATH}-Variable entscheidend.
Diese ist dafür verantwortlich, dass Programme in der aktuellen \gls{bash}-Session gefunden werden.\\
Anschließend erfolgt der Wechsel in die erstellte Verzeichnisstruktur.
Die Syntax von chroot ermöglicht es, nach dem Wechsel in das neue Wurzelverzeichnis, einen \gls{bash}-Befehl auszuführen.
In diesem werden alle -- für \gls{afl} relevanten -- Umgebungsvariablen gesetzt und mit passender Instrumentierung gestartet.
%! Author = chaorn
%! Date = 21.02.24
\lstdefinestyle{chroot}{
    showstringspaces=false
}

\begin{lstlisting}[language=bash, style=chroot, caption={Wechseln in ein anderes Wurzelverzeichnis mit chroot und ausführen von AFL},label={lst:chroot-fuzzen}]
$ sudo chroot root/
    bash -c 'export AFL_DEBUG=1; export AFL_PRELOAD=./sockfuzz.so;
    export QEMU_LD_PREFIX=/;
    export PATH="/usr/gnu/bin:/usr/local/sbin:/usr/local/bin:/bin:/sbin:/usr/bin:.";
    export QT_X11_NO_MITSHM=1; export DISPLAY=:10;
    ./afl-fuzz -Q -i in -o out -t 50000 -- /app/mmapp @@'
\end{lstlisting}
