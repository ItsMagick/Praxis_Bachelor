%! Author = charon
%! Date = 2/8/24

\subsection{Aufbau der Testumgebung mit bwrap}\label{subsec: bwrap}
Bubblewrap ist ein Low-Level Sandboxing-Tool.
Es ermöglicht dem Nutzer, eine containerisierte Laufzeitumgebung ähnlich wie Docker, zu erstellen.
Der Unterschied zu Docker ist, dass Bubblewrap~\cite{bwrap} (oder kurz \textit{bwrap}) es ermöglicht, auch unprivilegierten Benutzern
Container -- mithilfe von Linux Kernel-User-Namespaces -- zur Verfügung zu stellen.
Der Vorteil bei der Umsetzung einer Testumgebung mit einem Container liegt darin, dass es unwahrscheinlicher ist,
das Host-System fälschlicherweise umzukonfigurieren.
Wie bereits in Abschnitt~\ref{subsec: chroot} angesprochen, müssen auch mit bwrap die entsprechenden Verzeichnisse gemountet werden,
sodass das Binary fehlerfrei funktioniert.
Das muss jedoch nicht wie bei chroot auf dem Host-System erfolgen, sondern in einer Containerumgebung.
Bwrap erstellt bei Aufruf zuerst ein leeres \gls{fs}.
Danach können alle benötigten Verzeichnisse in den Container gemountet werden~\ref{lst:bwrap-env}.
Diese Methode hat sich als die beste Methode für das hardwareunabhängige Testen erwiesen.
Mit der Isolation der Applikation über Kernel-Namespaces bietet \gls{bwrap} mit einer einfachen Syntax eine sichere Möglichkeit,
Applikationen in eine Testumgebung zu packen.
Aufgrund der Containerisierung ist es im Gegensatz zu chroot nur erschwert möglich, das eigene Betriebssystem mit dem falschen Mounten
von Verzeichnissen zu beschädigen.
Auch hat man bei der Konfiguration und Inbetriebnahme der Applikation keinen Overhead der Virtualisierung einer kompletten Plattform
wie mit Docker, was bwrap zu einem geeigneten Tool zur Implementierung einer leichtgewichtigen Testumgebung macht.