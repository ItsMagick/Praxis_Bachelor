%! Author = charon
%! Date = 2/8/24
\subsection{Grey-Box-Fuzzing unter AFL++}\label{subsec:greybox-fuzzing}
Ein Fuzzer kann in die Kategorien \textit{White-Box-Fuzzer}, \textit{Black-Box-Fuzzer} und \textit{Grey-Box-Fuzzer} eingeteilt werden.
Diese Einteilung ergibt sich aus der Informationsgewinnung des zu untersuchenden Programms.\\
\linebreak
Ein White-Box-Fuzzer nutzt Informationen von vorhandenem Quellcode und der dessen Logik.
Dadurch ist es mit einem White-Box-Fuzzer möglich, sehr komplexe Schwachstellen innerhalb eines Programms zu finden. \\
Ein Black-Box-Fuzzer hingegen ist ein Fuzzer, der keine Verwendung des enthaltenen Quellcodes und der darin enthaltenen
Logik macht.
Er prüft lediglich, ob das zu untersuchende Programm den von ihm verwendeten Input akzeptiert\cite{iot-fuzzing}. \\
\gls{aflpp} ist ein Werkzeug, das ursprünglich dazu vorgesehen ist, eigenen Code an jeder Zweigstelle (branch) im
Programmcode eines zu untersuchenden Programms mithilfe des vorhandenen Quellcodes einzufügen.
Diesen Prozess nennt man Instrumentation.
Durch diesen Schritt kann die erreichte Tiefe (\textit{code coverage}) eines Inputs verfolgt werden~\cite{afl-instr}.
Dieser Schritt wird bereits beim Kompilieren des Programms mithilfe eines angepassten Compilers von AFL ausgeführt.
Da \gls{aflpp} die Informationen der erreichten Codepfade verwendet, um die Qualität des Inputs einzuschätzen,
wird der Fuzzer als Grey-Box-Fuzzer eingestuft.\\
Da in diesem Projekt eine Applikation untersucht wird, bei der kein Quellcode vorhanden ist, funktioniert ~\gls{aflpp}
anders.
\gls{aflpp} fügt während der Laufzeit einer emulierten Applikation Instruktionen zur Instrumentierung hinzu.
Hierbei werden, ähnlich wie beim zuvor angesprochenen Ansatz, die basic blocks auf Byte-Ebene betrachtet.
Dazu wird der Bytecode nicht angepasst.
Stattdessen wird jedes Byte jeder Vergleichsinstruktion mithilfe von
Hooks, die in \gls{aflpp} implementiert sind, verglichen.
Dadurch wird ermittelt, ob ein neuer Pfad im Programm gefunden wurde~\cite[vgl.][6]{AFLplusplus-Woot20}.
