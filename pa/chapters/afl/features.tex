%! Author = charon
%! Date = 2/8/24

\subsection{Features von AFL++}\label{subsec:features-von-afl++}
\gls{afl} ist ein sehr umfangreiches Werkzeug zum Testen der Stabilität von Programmen.
Das volle Potenzial des Fuzzers wurde im Rahmen dieser Arbeit nicht ausgeschöpft, da es nicht den Anforderungen der Arbeit
entspricht.
Jedoch wurden einige Modi und Werkzeuge von \gls{afl} verwendet.
Die in dieser Arbeit verwendeten Modi und Techniken von~\gls{afl} werden im Folgenden erläutert.
Zuerst wird ein Überblick über den~\gls{qemu}-Mode zur allgemeinen Instrumentierung des zu untersuchenden Programms gegeben.
Im Anschluss werden die Technologien und Ansätze des Fuzzings einer Netzwerkapplikation näher beschrieben, um
ein grobes Verständnis der Aufgabenstellung zu schaffen.
\subsubsection{QEMU-Mode zur Instrumentation von Binärdateien}
Wie bereits in der Sektion~\ref{subsec:afl} beschrieben, werden beim Fuzzen des Zielbinarys die Instrumentationsinstruktionen
für~\gls{afl} zur Laufzeit des Programms bereitgestellt.
Für das Untersuchen eines Binarys, welches eine andere Architektur als das Hostsystem besitzt, stellt \gls{afl}
einen Modus zum Emulieren eines anderen Systems zur Verfügung.
Dieser Modus in \gls{afl} wird \textit{\gls{qemu}-Mode} genannt.\\
\gls{qemu} ist ein Open-Source Emulations- und Virtualisierungsprogramm.
Die in \gls{qemu} enthaltenen Emulationen beschränken sich auf zwei Modi~\cite{qemu}:
\begin{itemize}
    \item System Emulation
    \item User Mode Emulation
\end{itemize}
In dem \textit{System Emulation} Modus wird ein komplettes System als virtualisiertes Modell bereitgestellt.
Dazu gehören physische Komponenten wie \gls{cpu} und physischer Speicher.
Die Besonderheit dieses Modus ist, dass die \gls{cpu} auch direkt mit dem Hostsystem von einem Hypervisor verwendet werden
kann. \\
Bei der \textit{User Mode Emulation} wird die \gls{cpu} immer vollständig emuliert.
Dieser Modus ermöglicht, es ein Programm platformunabhängig in einer virtualisierten Umgebung, zu starten.
\gls{afl} verwendet zur Untersuchung eines Binarys ohne Zugriff auf den dazugehörigen Quellcode eine eigene, leicht
angepasste Version der User Mode Emulation~\cite{afl-qemu-how}.\\
\linebreak
Die von \gls{afl} benötigten Daten der Codepfadabdeckung werden über einen zwischen \gls{qemu} und dem Fuzzer geteilten
Speicher an den Fuzzer übergeben.
Dabei ist \gls{afl} für die Verwaltung des geteilten Speichers verantwortlich.
Die von \gls{afl} angepasste Version von \gls{qemu} ist für das Sammeln und Befüllen des geteilten Speichers verantwortlich~\cite{afl-qemu-coverage}.
\subsubsection{Fuzzing einer Netzwerkapplikation}\label{subsubsec:fuzzing-netzwerk-app}
Das Fuzzen von Netzwerkapplikationen ist mit~\gls{afl} nicht vorgesehen.
Benutzt man hierbei einen herkömmlichen Netzwerkchannel im Gegensatz zu der vorgesehenen Herangehensweise aus der Sektion\ref{subsec:afl},
so muss mit Geschwindigkeitsverlusten von bis zu einem Faktor von 20 gerechnet werden.
Um das Fuzzen von Netzwerkapplikationen zu ermöglichen, müssen einige Tricks verwendet werden.
Dazu gehört die Anpassung des Sourcecodes, sodass die Applikation Eingaben über stdin oder über Filedeskriptoren statt über Netzwerksockets liest.
\linebreak
Möchte man dies realisieren, so gibt es die Möglichkeit, mithilfe einer \textit{preload library} den bereits
implementierten Programmcode zu überschreiben.
Von~\gls{afl} existiert bereits eine rudimentäre Implementierung einer solchen Bibliothek~\cite{afl-best-practice}, die für jedes Binary
überschrieben werden muss.
Dieser Bibliothek wurde in dieser Arbeit erweitert und auf die Anforderungen des zu untersuchenden Programmes
angepasst.
Die genauere Umsetzung der preload library wird in Abschnitt ~\ref{subsec: desocketing} beschrieben.