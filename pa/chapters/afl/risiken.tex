%! Author = charon
%! Date = 2/8/24
\subsection{Risiken}\label{subsec:risiken}
Das Fuzzen von Applikationen bringt auch einige Risiken mit sich, die vor dem Start der Kampagne berücksichtigt werden sollten.
Es handelt sich dabei um sehr leistungsintensive Anweisungen.
Darunter fallen das Transpilieren -- also Übersetzen -- von Instruktionen auf eine andere Architektur,
das Mutieren der Eingaben und das Verfolgen der abgedeckten Programmpfade, sowie das Lesen von Eingaben und das Schreiben von Output.
Insbesondere bei der Parallelisierung der Kampagne und der Ausführung dieser auf mehreren \gls{cpu}-Kernen ist auf die Temperatur
des Testsystems zu achten.
Die folgenden Absätze basieren auf der Dokumentation~\cite{fuzzing-risks} von \gls{aflpp} und wurden zur besseren
Veranschaulichung erweitert. \\
\linebreak
Lese- und Schreibinstruktionen, die eine große Belastung eines Systems darstellen, und das damit verbundene Caching
können einen wesentlichen Faktor der Erhitzung der \gls{cpu} sein.
Dies kann zur Drosselung der Leistung des Systems als Selbstschutzmechanismus und im Extremfall zum
Systemabsturz oder zur Beschädigung der Hardware führen.
Es wird daher empfohlen, eine spezialisierte Langzeitkampagne auf einem System mit ausreichender Kühlung durchzuführen,
damit die im System verbauten Komponenten keinen Schaden davontragen. \\
\linebreak
Darüber hinaus besteht die Gefahr, dass es bei einer lang anhaltenden Kampagne zum Datenverlust kommen kann.
Dieses Risiko wird durch die häufige Generierung vieler Dateien und deren Abspeicherung verursacht.
Außerdem können viele Logdateien, wie Absturzprotokolle oder \textit{core dumps}, zu einem hohen Speicherverbrauch während
der Kampagne führen.
Wenn dabei nicht genug Speicherplatz zur Verfügung steht, kann es passieren, dass Daten von dem von \gls{afl} erzeugten
Output überschrieben werden.
Daraus folgt, dass eine Langzeitkampagne -- über mehrere Tage -- nicht auf Systemen durchgeführt werden sollte, bei denen
ein resultierender Datenverlust nicht tragbar ist.
Daher sollte man immer feststellen, dass genug Speicherplatz für eine lange Fuzzing-Kampagne zur Verfügung steht. \\
\linebreak
Schließlich trägt das Fuzzing einer Applikation dazu bei, die Lebensdauer einer physischen Speichereinheit wie einer~\gls{hdd}
oder einer~\gls{ssd}, erheblich zu reduzieren.
Dieser Beschädigung der Hardware ist auf sehr intensiven Lese- und Schreibaktivität auf dem Speichermedium zurückzuführen.
Bei der Generierung von mutierten Daten und dem Lesen und Schreiben dieser Daten auf ein Speichermedium geschehen im Verlauf
einer Kampagne Milliarden von Lese- und Schreibzyklen.
Bei dem häufigen Schreiben und Lesen von Daten eines Speichermediums verschleißen die darin eingebauten Teile.
Eine Möglichkeit, dem Verschleiß eines Speichermediums im gewissen Umfang entgegenzuwirken, ist das Caching der temporär erzeugten Daten
im~\gls{ram}. \\
\linebreak
Unter Abwägung aller Risiken ist es daher ratsam, anspruchsvolle Fuzzing-Kampagnen auf einem dedizierten System laufen zu lassen.
Wenn diese Möglichkeit nicht besteht, sollte darauf geachtet werden, dass eine solche Kampagne nur für eine ausreichend kurze Zeit läuft.
Des Weiteren sollte immer eine gute Backup-Strategie zur Hand sein, um einem möglichen Datenverlust entgegenzuwirken.