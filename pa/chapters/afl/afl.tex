%! Author = charon
%! Date = 2/1/24

\section{American Fuzzy Lop}\label{sec:american-fuzzy-loop}
Bei der Wahl eines Fuzzers gilt es auf folgende Merkmale zu achten:
\begin{itemize}
    \item Kompatibilität mit verschiedenen ~\gls{cpu}-Architekturen
    \item Art des Fuzzers
    \item Strategie des Fuzzers
    \item Features des Fuzzers
    \item Benutzbarkeit des Fuzzers
\end{itemize}
Gerade bei~\gls{iot}-Programmen ist die Kompatibilität mit verschiedenen \gls{cpu}-Architekturen ausschlaggebend.
Im Fall dieser Arbeit handelt es sich um ein Netzwerk-Binary mit einer \gls{arm}v7-Architektur.
\gls{aflpp} stellt die in diesem Projekt benötigten Werkzeuge bereits zur Verfügung.
Die genauere Funktionsweise, Strategie, Kompatibilität sowie die entscheidenden Features von~\gls{aflpp} werden im Folgenden näher betrachtet.
%! Author = charon
%! Date = 2/8/24
\subsection{Grey-Box-Fuzzing unter AFL++}\label{subsec:greybox-fuzzing}
Ein Fuzzer kann in die Kategorien \textit{White-Box-Fuzzer}, \textit{Black-Box-Fuzzer} und \textit{Grey-Box-Fuzzer} eingeteilt werden.
Diese Einteilung ergibt sich aus der Informationsgewinnung des zu untersuchenden Programms.\\
\linebreak
Ein White-Box-Fuzzer nutzt Informationen von vorhandenem Quellcode und der dessen Logik.
Dadurch ist es mit einem White-Box-Fuzzer möglich, sehr komplexe Schwachstellen innerhalb eines Programms zu finden. \\
Ein Black-Box-Fuzzer hingegen ist ein Fuzzer, der keine Verwendung des enthaltenen Quellcodes und der darin enthaltenen
Logik macht.
Er prüft lediglich, ob das zu untersuchende Programm den von ihm verwendeten Input akzeptiert\cite{iot-fuzzing}. \\
\gls{aflpp} ist ein Werkzeug, das ursprünglich dazu vorgesehen ist, eigenen Code an jeder Zweigstelle (branch) im
Programmcode eines zu untersuchenden Programms mithilfe des vorhandenen Quellcodes einzufügen.
Diesen Prozess nennt man Instrumentation.
Durch diesen Schritt kann die erreichte Tiefe (\textit{code coverage}) eines Inputs verfolgt werden~\cite{afl-instr}.
Dieser Schritt wird bereits beim Kompilieren des Programms mithilfe eines angepassten Compilers von AFL ausgeführt.
Da \gls{aflpp} die Informationen der erreichten Codepfade verwendet, um die Qualität des Inputs einzuschätzen,
wird der Fuzzer als Grey-Box-Fuzzer eingestuft.\\
Da in diesem Projekt eine Applikation untersucht wird, bei der kein Quellcode vorhanden ist, funktioniert ~\gls{aflpp}
anders.
\gls{aflpp} fügt während der Laufzeit einer emulierten Applikation Instruktionen zur Instrumentierung hinzu.
Hierbei werden, ähnlich wie beim zuvor angesprochenen Ansatz, die basic blocks auf Byte-Ebene betrachtet.
Dazu wird der Bytecode nicht angepasst.
Stattdessen wird jedes Byte jeder Vergleichsinstruktion mithilfe von
Hooks, die in \gls{aflpp} implementiert sind, verglichen.
Dadurch wird ermittelt, ob ein neuer Pfad im Programm gefunden wurde~\cite[vgl.][6]{AFLplusplus-Woot20}.

%! Author = charon
%! Date = 2/8/24

\subsection{Auswahl des Korpus der Fuzzing Kampagne}\label{subsec:corpus}

Der Korpus -- die von~\gls{afl} und~\gls{aflpp} verwendeten Inputs -- spielt eine zentrale Rolle in der Fuzzing-Kampagne.
Er besteht aus Daten, die von dem zu untersuchenden Binary gelesen und verarbeitet werden.
Sie dienen als Anhaltspunkt für~\gls{aflpp}, um das Programm zu starten und erfolgreich zu terminieren und somit möglichst viele Programmpfade abzulaufen.
Der Korpus kann, je nach Anwendungsfall, unterschiedlich aussehen.
Beispielsweise erwartet ein Programm wie Adobe Acrobat Reader, Dateien mit der Dateiendung \textit{.pdf}.
Die interne Struktur der darin enthaltenen Daten muss in~\gls{pdf}-Syntax~\ref{lst:example.pdf} vorliegen.
Ein~\gls{cli}-Programm erwartet andererseits Daten, die ihm bereits beim Start übergeben werden müssen (siehe Listing~\ref{lst:stdin_program.c}).
\linebreak
Der initiale Korpus beim Starten des Binarys mit~\gls{aflpp} kann aus mindestens einer Datei bestehen.
Die optimale Dateigröße beträgt unter einem~\gls{kb}.
Mehrere Dateien im Korpus sollen nur vorhanden sein, wenn diese auch verschiedene Programmpfade - also andere Zustände - des Programms erreichen.\\
\linebreak
Das Netzwerkprotokoll der zu untersuchenden Applikation verarbeitet Bytes.
Hierbei können die an das Protokoll gesendeten Bytes in ihrer Länge variieren.
So steht der Befehl \texttt{\textbackslash x02\textbackslash x00\textbackslash x00\textbackslash x00\textbackslash x00\textbackslash x02}
für das Hochfahren des Gerätes.
Bei solchen Standardbefehlen stehen alle Bytes bereits fest und müssen nicht manuell angepasst werden.
Es existieren ebenfalls variable Befehle.
Der Befehl \texttt{\textbackslash x03\textbackslash x9A\textbackslash x00\textbackslash x00\textbackslash x01\textbackslash<Var1>\textbackslash<CKS>}
ist für die Anfrage der ~\ce{CO2} Ersparnisse einer speziellen Peripherie verantwortlich.
Beim Senden dieses Befehls müssen vom Benutzer händisch eingetragene Werte mitgegeben werden.
Das Symbol \texttt{<Var1>} steht für ein Byte und das Symbol \texttt{<CKS>} steht für die Prüfsumme.
Sie besteht aus dem letzten Byte der Quersumme aller zu sendenden Bytes.
In dem Protokoll, welches diese Befehle verarbeitet, existiert eine \textit{state machine}, welche das Programm anhand der gelieferten
Daten in einen definierten Zustand versetzt. \\
\linebreak\linebreak
Mit dem Wissen ist ein Befehl in zwei Teile einteilbar.
%! Author = chaorn
%! Date = 08.03.24

\begin{lstlisting}[caption={Formaler Aufbau eines Befehls des Binärprotokolls bestehend aus einem Befehlsrumpf und der dazugehörigen
Prüfsumme.},label={lst:command-structure}]
<Byte01> <Byte02> <Byte03> <Byte04> <Byte05> ... <CKS>
\end{lstlisting}
Der zu sendende Befehl besteht mindestens aus fünf Bytes, gefolgt von der bereits beschriebenen Prüfsumme.


%! Author = charon
%! Date = 2/8/24

\subsection{Features von AFL++}\label{subsec:features-von-afl++}
\gls{afl} ist ein sehr umfangreiches Werkzeug zum Testen der Stabilität von Programmen.
Das volle Potenzial des Fuzzers wurde im Rahmen dieser Arbeit nicht ausgeschöpft, da es nicht den Anforderungen der Arbeit
entspricht.
Jedoch wurden einige Modi und Werkzeuge von \gls{afl} verwendet.
Die in dieser Arbeit verwendeten Modi und Techniken von~\gls{afl} werden im Folgenden erläutert.
Zuerst wird ein Überblick über den~\gls{qemu}-Mode zur allgemeinen Instrumentierung des zu untersuchenden Programms gegeben.
Im Anschluss werden die Technologien und Ansätze des Fuzzings einer Netzwerkapplikation näher beschrieben, um
ein grobes Verständnis der Aufgabenstellung zu schaffen.
\subsubsection{QEMU-Mode zur Instrumentation von Binärdateien}
Wie bereits in der Sektion~\ref{subsec:afl} beschrieben, werden beim Fuzzen des Zielbinarys die Instrumentationsinstruktionen
für~\gls{afl} zur Laufzeit des Programms bereitgestellt.
Für das Untersuchen eines Binarys, welches eine andere Architektur als das Hostsystem besitzt, stellt \gls{afl}
einen Modus zum Emulieren eines anderen Systems zur Verfügung.
Dieser Modus in \gls{afl} wird \textit{\gls{qemu}-Mode} genannt.\\
\gls{qemu} ist ein Open-Source Emulations- und Virtualisierungsprogramm.
Die in \gls{qemu} enthaltenen Emulationen beschränken sich auf zwei Modi~\cite{qemu}:
\begin{itemize}
    \item System Emulation
    \item User Mode Emulation
\end{itemize}
In dem \textit{System Emulation} Modus wird ein komplettes System als virtualisiertes Modell bereitgestellt.
Dazu gehören physische Komponenten wie \gls{cpu} und physischer Speicher.
Die Besonderheit dieses Modus ist, dass die \gls{cpu} auch direkt mit dem Hostsystem von einem Hypervisor verwendet werden
kann. \\
Bei der \textit{User Mode Emulation} wird die \gls{cpu} immer vollständig emuliert.
Dieser Modus ermöglicht, es ein Programm platformunabhängig in einer virtualisierten Umgebung, zu starten.
\gls{afl} verwendet zur Untersuchung eines Binarys ohne Zugriff auf den dazugehörigen Quellcode eine eigene, leicht
angepasste Version der User Mode Emulation~\cite{afl-qemu-how}.\\
\linebreak
Die von \gls{afl} benötigten Daten der Codepfadabdeckung werden über einen zwischen \gls{qemu} und dem Fuzzer geteilten
Speicher an den Fuzzer übergeben.
Dabei ist \gls{afl} für die Verwaltung des geteilten Speichers verantwortlich.
Die von \gls{afl} angepasste Version von \gls{qemu} ist für das Sammeln und Befüllen des geteilten Speichers verantwortlich~\cite{afl-qemu-coverage}.
\subsubsection{Fuzzing einer Netzwerkapplikation}\label{subsubsec:fuzzing-netzwerk-app}
Das Fuzzen von Netzwerkapplikationen ist mit~\gls{afl} nicht vorgesehen.
Benutzt man hierbei einen herkömmlichen Netzwerkchannel im Gegensatz zu der vorgesehenen Herangehensweise aus der Sektion\ref{subsec:afl},
so muss mit Geschwindigkeitsverlusten von bis zu einem Faktor von 20 gerechnet werden.
Um das Fuzzen von Netzwerkapplikationen zu ermöglichen, müssen einige Tricks verwendet werden.
Dazu gehört die Anpassung des Sourcecodes, sodass die Applikation Eingaben über stdin oder über Filedeskriptoren statt über Netzwerksockets liest.
\linebreak
Möchte man dies realisieren, so gibt es die Möglichkeit, mithilfe einer \textit{preload library} den bereits
implementierten Programmcode zu überschreiben.
Von~\gls{afl} existiert bereits eine rudimentäre Implementierung einer solchen Bibliothek~\cite{afl-best-practice}, die für jedes Binary
überschrieben werden muss.
Dieser Bibliothek wurde in dieser Arbeit erweitert und auf die Anforderungen des zu untersuchenden Programmes
angepasst.
Die genauere Umsetzung der preload library wird in Abschnitt ~\ref{subsec: desocketing} beschrieben.
%! Author = charon
%! Date = 2/8/24
\subsection{Risiken}\label{subsec:risiken}
Das Fuzzen von Applikationen bringt auch einige Risiken mit sich, die vor dem Start der Kampagne berücksichtigt werden sollten.
Es handelt sich dabei um sehr leistungsintensive Anweisungen.
Darunter fallen das Transpilieren -- also Übersetzen -- von Instruktionen auf eine andere Architektur,
das Mutieren der Eingaben und das Verfolgen der abgedeckten Programmpfade, sowie das Lesen von Eingaben und das Schreiben von Output.
Insbesondere bei der Parallelisierung der Kampagne und der Ausführung dieser auf mehreren \gls{cpu}-Kernen ist auf die Temperatur
des Testsystems zu achten.
Die folgenden Absätze basieren auf der Dokumentation~\cite{fuzzing-risks} von \gls{aflpp} und wurden zur besseren
Veranschaulichung erweitert. \\
\linebreak
Lese- und Schreibinstruktionen, die eine große Belastung eines Systems darstellen, und das damit verbundene Caching
können einen wesentlichen Faktor der Erhitzung der \gls{cpu} sein.
Dies kann zur Drosselung der Leistung des Systems als Selbstschutzmechanismus und im Extremfall zum
Systemabsturz oder zur Beschädigung der Hardware führen.
Es wird daher empfohlen, eine spezialisierte Langzeitkampagne auf einem System mit ausreichender Kühlung durchzuführen,
damit die im System verbauten Komponenten keinen Schaden davontragen. \\
\linebreak
Darüber hinaus besteht die Gefahr, dass es bei einer lang anhaltenden Kampagne zum Datenverlust kommen kann.
Dieses Risiko wird durch die häufige Generierung vieler Dateien und deren Abspeicherung verursacht.
Außerdem können viele Logdateien, wie Absturzprotokolle oder \textit{core dumps}, zu einem hohen Speicherverbrauch während
der Kampagne führen.
Wenn dabei nicht genug Speicherplatz zur Verfügung steht, kann es passieren, dass Daten von dem von \gls{afl} erzeugten
Output überschrieben werden.
Daraus folgt, dass eine Langzeitkampagne -- über mehrere Tage -- nicht auf Systemen durchgeführt werden sollte, bei denen
ein resultierender Datenverlust nicht tragbar ist.
Daher sollte man immer feststellen, dass genug Speicherplatz für eine lange Fuzzing-Kampagne zur Verfügung steht. \\
\linebreak
Schließlich trägt das Fuzzing einer Applikation dazu bei, die Lebensdauer einer physischen Speichereinheit wie einer~\gls{hdd}
oder einer~\gls{ssd}, erheblich zu reduzieren.
Dieser Beschädigung der Hardware ist auf sehr intensiven Lese- und Schreibaktivität auf dem Speichermedium zurückzuführen.
Bei der Generierung von mutierten Daten und dem Lesen und Schreiben dieser Daten auf ein Speichermedium geschehen im Verlauf
einer Kampagne Milliarden von Lese- und Schreibzyklen.
Bei dem häufigen Schreiben und Lesen von Daten eines Speichermediums verschleißen die darin eingebauten Teile.
Eine Möglichkeit, dem Verschleiß eines Speichermediums im gewissen Umfang entgegenzuwirken, ist das Caching der temporär erzeugten Daten
im~\gls{ram}. \\
\linebreak
Unter Abwägung aller Risiken ist es daher ratsam, anspruchsvolle Fuzzing-Kampagnen auf einem dedizierten System laufen zu lassen.
Wenn diese Möglichkeit nicht besteht, sollte darauf geachtet werden, dass eine solche Kampagne nur für eine ausreichend kurze Zeit läuft.
Des Weiteren sollte immer eine gute Backup-Strategie zur Hand sein, um einem möglichen Datenverlust entgegenzuwirken.