%! Author = charon
%! Date = 2/8/24

\subsection{Auswahl des Korpus der Fuzzing Kampagne}\label{subsec:corpus}

Der Korpus -- die von~\gls{afl} und~\gls{aflpp} verwendeten Inputs -- spielt eine zentrale Rolle in der Fuzzing-Kampagne.
Er besteht aus Daten, die von dem zu untersuchenden Binary gelesen und verarbeitet werden.
Sie dienen als Anhaltspunkt für~\gls{aflpp}, um das Programm zu starten und erfolgreich zu terminieren und somit möglichst viele Programmpfade abzulaufen.
Der Korpus kann, je nach Anwendungsfall, unterschiedlich aussehen.
Beispielsweise erwartet ein Programm wie Adobe Acrobat Reader, Dateien mit der Dateiendung \textit{.pdf}.
Die interne Struktur der darin enthaltenen Daten muss in~\gls{pdf}-Syntax~\ref{lst:example.pdf} vorliegen.
Ein~\gls{cli}-Programm erwartet andererseits Daten, die ihm bereits beim Start übergeben werden müssen (siehe Listing~\ref{lst:stdin_program.c}).
\linebreak
Der initiale Korpus beim Starten des Binarys mit~\gls{aflpp} kann aus mindestens einer Datei bestehen.
Die optimale Dateigröße beträgt unter einem~\gls{kb}.
Mehrere Dateien im Korpus sollen nur vorhanden sein, wenn diese auch verschiedene Programmpfade - also andere Zustände - des Programms erreichen.\\
\linebreak
Das Netzwerkprotokoll der zu untersuchenden Applikation verarbeitet Bytes.
Hierbei können die an das Protokoll gesendeten Bytes in ihrer Länge variieren.
So steht der Befehl \texttt{\textbackslash x02\textbackslash x00\textbackslash x00\textbackslash x00\textbackslash x00\textbackslash x02}
für das Hochfahren des Gerätes.
Bei solchen Standardbefehlen stehen alle Bytes bereits fest und müssen nicht manuell angepasst werden.
Es existieren ebenfalls variable Befehle.
Der Befehl \texttt{\textbackslash x03\textbackslash x9A\textbackslash x00\textbackslash x00\textbackslash x01\textbackslash<Var1>\textbackslash<CKS>}
ist für die Anfrage der ~\ce{CO2} Ersparnisse einer speziellen Peripherie verantwortlich.
Beim Senden dieses Befehls müssen vom Benutzer händisch eingetragene Werte mitgegeben werden.
Das Symbol \texttt{<Var1>} steht für ein Byte und das Symbol \texttt{<CKS>} steht für die Prüfsumme.
Sie besteht aus dem letzten Byte der Quersumme aller zu sendenden Bytes.
In dem Protokoll, welches diese Befehle verarbeitet, existiert eine \textit{state machine}, welche das Programm anhand der gelieferten
Daten in einen definierten Zustand versetzt. \\
\linebreak\linebreak
Mit dem Wissen ist ein Befehl in zwei Teile einteilbar.
%! Author = chaorn
%! Date = 08.03.24

\begin{lstlisting}[caption={Formaler Aufbau eines Befehls des Binärprotokolls bestehend aus einem Befehlsrumpf und der dazugehörigen
Prüfsumme.},label={lst:command-structure}]
<Byte01> <Byte02> <Byte03> <Byte04> <Byte05> ... <CKS>
\end{lstlisting}
Der zu sendende Befehl besteht mindestens aus fünf Bytes, gefolgt von der bereits beschriebenen Prüfsumme.

