%! Author = charon
%! Date = 2/7/24

\subsection{Motivation}\label{subsec:motivation}
Diese Arbeit soll als Anleitung zum Entwickeln einer Fuzzing-Kampagne mit \gls{afl} dienen.
Mit der immer weiter ansteigenden Nutzung ovn \gls{iot}-Geräten im Alltag muss in dem Bereich der Netzwerksicherheit
von untereinander verbundenen Geräten ein besonderes Augenmerk geworfen werden.
Die derzeitige Schätzung der aktuell mit dem Internet verbundenen \gls{iot}-Geräte liegt bei ca.\ 3,5 Milliarden Geräten
im Jahr 2023~\cite{iot-statistik}. \\
Die derzeit führende Sprache für die Entwicklung der Firmware solcher Geräte ist die Programmiersprache C\@.
Auch trotz bereits existierender Frameworks und Programmiersprachen, wie Rust oder Go wird die Programmiersprache
C oft den anderen Optionen vorgezogen und zählt somit zu einer der meistverwendeten Programmiersprachen~\cite{tiobe-programming-trends}.
Das ist der Leichtgewichtigkeit und vor allem Geschwindigkeit der Programmiersprache geschuldet.
Diese zwei Faktoren spielen eine tragende Rolle bei der Implementierung von Software für \gls{iot}-Geräte aufgrund
der sehr limitierten Ressourcen, die solche Geräte mit sich bringen.
Bei der Auswahl der passenden Programmiersprache werden jedoch wichtige Kriterien, wie die Sicherheit einer Programmiersprache
weniger in Erwägung gezogen.
In Programmen, die in der Programmiersprache C geschrieben wurden, werden die häufigsten schwerwiegenden Schwachstellen
mit einer Bewertung eines \gls{cvss} Score von mindestens 7 gefunden.
Dabei ist diese Programmiersprache für über \SI{50}{\percent} aller öffentlich dokumentierten Schwachstellen verantwortlich~\cite{most-secure-programming}.
Das \gls{cvss} ist ein Bewertungssystem, das die Schwere von Sicherheitslücken kategorisiert.
Die dabei verwendete Skala der \gls{cvss} Version 3.0 ordnet hierbei Schwachstellen mit einem Score von 0.0 als
gering, 0.1-3.9 als niedrig, 4.0-6.9 als medium, 7.0-8.9 als hoch und 9.0-10.0 als kritisch ein~\cite{cvss}.
Als kritisch eingestufte Schwachstellen sind in der Regel Schwachstellen, welche es einem Angreifer erlauben, \gls{rce}
auszuführen und somit die Möglichkeit besteht, unbefugt Informationen und Daten über ein System zu erlangen.
Zu der Klasse der meistverbreiteten Schwachstellen dieser Art gehören Memory Corruption Schwachstellen, welche es ermöglichen
auf unbefugten Speicher eines Systems zugreifen zu können.
Genau diese Schwachstellen beispielsweise in Form eines Buffer Overflows gehören zu den meist gefundenen Schwachstellen in
Programmen, welche auf der Programmiersprache C basieren.
Aufgrund der weiten Verbreitung und gewinnenden Bedeutung von \gls{iot}-Geräten im alltäglichen Leben soll diese Arbeit
eine Möglichkeit darbieten, mithilfe von Fuzzing automatisiert Schwachstellen in diesen Geräten zu finden.
Mithilfe solcher Methodiken soll in Zukunft die Ausnutzung schwerwiegender Schwachstellen vorgebeugt werden.\\
\linebreak
Das Fuzzing von Netzwerkapplikationen ist derzeit in den Startlöchern und eine weitaus unerforschte Disziplin.
Dabei zu beachten ist, dass die Performance der Protokoll-Fuzzer weitaus unter der, der Applikations-Fuzzer liegt.
Somit bietet diese Arbeit einen alternativen Ansatz zum Fuzzen von Netzwerkprotokollen auf Applikationsebene, um diesem
Performanceverlust entgegenzuwirken.