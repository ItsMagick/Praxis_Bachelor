%! Author = charon
%! Date = 2/7/24
\subsection{Zielsetzung}\label{subsec:zielsetzung}
Wie bereits erwähnt, soll diese Arbeit eine Möglichkeit offenlegen, wie ein Netzwerkprotokoll auf Applikationsebene
gefuzzt werden kann, sodass die Leistung des Fuzzers nicht darunter leidet.
Das bedeutet nicht nur, dass hier ein Weg des Protokoll-Fuzzings beschrieben wird, sondern auch die generelle Herangehensweise
an eine Fuzzing-Kampagne und die tieferen Hintergründe des \gls{afl} Fuzzers genauer erläutert werden. \\
\linebreak
Im Laufe dieser Arbeit soll die Vorgehensweise, mit der die Fuzzing-Kampagne umgesetzt wurde, dargestellt werden.
Des Weiteren werden verwendete Technologien genau erklärt und deren Rolle im Fuzzing-Prozess beleuchtet werden.
Dabei soll vor allem auf die Funktionsweise und die Interaktion von den verwendeten Technologien auf die Applikation
zum allgemeinen Verständnis der Applikation eingegangen werden.
Sobald die Technologien ausreichend erklärt wurden, folgt das Eingehen auf die genauere Implementierung und Instrumentierung
der zu untersuchenden Applikation.
Dieser Teil dient als Dokumentation zur tatsächlichen Umsetzung einer Fuzzing-Kampagne.