\section{Einleitung}\label{sec:einleitung}
Immer mehr gewinnen smarte Geräte Bedeutung in unser aller Alltag.
Die Verwendung solcher Geräte reicht von Glühlampen, welche sich an den Stromverbrauch des Endnutzers anpassen,
bis hin zu einem Beamer, welcher vielseitig und alltäglich in Firmen, Freizeitanlagen und auch Bildungsinstituten zum
Einsatz kommt.
Gerade bei diesen Geräten sollte, aufgrund der vielseitigen Ausnutzungsmöglichkeiten für beispielsweise \gls{ddos} Angriffe,
ein genaues Augenmerk auf Sicherheit gesetzt werden~\cite{iot-exploitations}. \\
Um solchen Gefahren vorzubeugen, führt die Forschungsgruppe \textit{Systems and Network Security} regelmäßige Schwachstellenscans durch.
Die Scans werden mithilfe eines angepassten Abbildes des Greenbone~\cite{greenbone} Schwachstellenscanners durchgeführt.
Nach einem der Scans ist Pahl -- einem wissenschaftlichen Mitarbeiter der Forschungsgruppe \gls{sns}\cite{sns} -- auf eine
bisher unentdeckte Schwachstelle im Web-Interface eines in der Hochschule eingesetzten \gls{iot}-Geräts gestoßen.
Hierbei handelt es sich um eine Path Traversal~\cite{path-traversal} Schwachstelle in der \gls{url} des Web-Interfaces eines
an der Hochschule verwendeten \gls{iot}-Geräts.
Genauere Details zu der Arbeit können zum Zeitpunkt der Verfassung dieses Dokuments, aufgrund einer noch nicht
abgeschlossenen \textit{responsible disclosure}, nicht preisgegeben werden.\\
\linebreak
Genau dieser Problematik widmet sich diese Arbeit, in der in den folgenden Kapiteln darauf eingegangen wird, wie es zur Findung von Schwachstellen
in der Firmware von \gls{iot}-Geräten kommt.
Sie dient als Anleitung für das methodische Testen von Firmware.
Die Herangehensweise hierbei ist das automatisierte Testen mithilfe eines Fuzzers, welcher durch Generierung von Input
Schwachstellen in einem Programm finden soll ~\cite{fuzzing}.
In dieser Arbeit wird insbesondere auf die Verwendung des Fuzzers \gls{aflpp} eingegangen.\\
\linebreak
Zuerst sollen kurz technische Hintergründe erläutert werden, um ein gewisses Grundverständnis der zum Einsatz gebrachten
Technologien aufzubauen.
Hierbei wird das Grundlegende Setup und die zugrunde liegende Firmware erklärt und eine grobe Übersicht über
\gls{afl}\cite{afl} gegeben.
In dieser Arbeit wird überwiegend die Erweiterung des Fuzzers \gls{aflpp} verwendet.
\gls{afl} wird nur als Referenz auf die grundlegende Funktionsweise des Fuzzers hergenommen.\\
\linebreak
Im Anschluss werden die genaueren Modi und Terminologien des \glstext{aflpp}~\cite{AFLplusplus-Woot20} geklärt, welche in dieser Arbeit Verwendung finden.
Dazu gehört, wie \gls{aflpp} mit der zu untersuchenden Applikation interagiert und kommuniziert.
Genauer eingegangen wird zudem auf das Fuzzing von Netzwerkapplikationen, welche den Tester vor Herausforderungen stellt, da diese Funktionalität nicht für \gls{aflpp} implementiert ist.
Ebenfalls werden die mit dem Fuzzing verbundenen Risiken angesprochen. \\
\linebreak
Daraufhin soll auf den Aufbau und die Implementierung der Testumgebung zur Analyse der Firmware für ein tieferes
Verständnis der Testumgebung eingegangen werden.
Hierbei wird auf die Funktionalitäten der Firmware eingegangen, welche einen Einblick in den Aufbau der Firmware und der Interaktion mit dem zugrunde liegenden
Betriebssystem geben.
Zudem wird erklärt, wie mit der Applikation interagiert werden kann.\\
\linebreak
Nachdem ein allgemeines Verständnis der verwendeten Technologien vermittelt wurde, wird die Durchführung des Fuzzing der Applikation beschrieben.
Dieser Abschnitt ist vor allem für unerfahrene Tester und Schwachstellenforscher, denen hier ein Einblick in die
Instrumentation und Modifikation zum Lesen von Dateien einer Applikation geboten wird.
Zudem werden Möglichkeiten zur Optimierung des Fuzzingprozesses sowie deren Anwendung auf eine laufende Fuzzing Kampagne offengelegt. \\
\linebreak
Zum Abschluss sollen noch die besonderen Hindernisse bei der Durchführung des Fuzzings genauer betrachtet werden.
Das soll vor allem bei der Umsetzung späterer Projekte dabei verhelfen, da hierbei auch auf allgemeine Problemstellungen eingegangen wird.
Anschließend daran soll ein Ausblick zu alternativen Umsetzungen des Fuzzings gegeben werden.
Abschließend soll das Projekt in seiner Gesamtheit reflektiert werden.

%! Author = charon
%! Date = 2/7/24

\subsection{Motivation}\label{subsec:motivation}
Diese Arbeit soll als Anleitung zum Entwickeln einer Fuzzing-Kampagne mit \gls{afl} dienen.
Mit der immer weiter ansteigenden Nutzung ovn \gls{iot}-Geräten im Alltag muss in dem Bereich der Netzwerksicherheit
von untereinander verbundenen Geräten ein besonderes Augenmerk geworfen werden.
Die derzeitige Schätzung der aktuell mit dem Internet verbundenen \gls{iot}-Geräte liegt bei ca.\ 3,5 Milliarden Geräten
im Jahr 2023~\cite{iot-statistik}. \\
Die derzeit führende Sprache für die Entwicklung der Firmware solcher Geräte ist die Programmiersprache C\@.
Auch trotz bereits existierender Frameworks und Programmiersprachen, wie Rust oder Go wird die Programmiersprache
C oft den anderen Optionen vorgezogen und zählt somit zu einer der meistverwendeten Programmiersprachen~\cite{tiobe-programming-trends}.
Das ist der Leichtgewichtigkeit und vor allem Geschwindigkeit der Programmiersprache geschuldet.
Diese zwei Faktoren spielen eine tragende Rolle bei der Implementierung von Software für \gls{iot}-Geräte aufgrund
der sehr limitierten Ressourcen, die solche Geräte mit sich bringen.
Bei der Auswahl der passenden Programmiersprache werden jedoch wichtige Kriterien, wie die Sicherheit einer Programmiersprache
weniger in Erwägung gezogen.
In Programmen, die in der Programmiersprache C geschrieben wurden, werden die häufigsten schwerwiegenden Schwachstellen
mit einer Bewertung eines \gls{cvss} Score von mindestens 7 gefunden.
Dabei ist diese Programmiersprache für über \SI{50}{\percent} aller öffentlich dokumentierten Schwachstellen verantwortlich~\cite{most-secure-programming}.
Das \gls{cvss} ist ein Bewertungssystem, das die Schwere von Sicherheitslücken kategorisiert.
Die dabei verwendete Skala der \gls{cvss} Version 3.0 ordnet hierbei Schwachstellen mit einem Score von 0.0 als
gering, 0.1-3.9 als niedrig, 4.0-6.9 als medium, 7.0-8.9 als hoch und 9.0-10.0 als kritisch ein~\cite{cvss}.
Als kritisch eingestufte Schwachstellen sind in der Regel Schwachstellen, welche es einem Angreifer erlauben, \gls{rce}
auszuführen und somit die Möglichkeit besteht, unbefugt Informationen und Daten über ein System zu erlangen.
Zu der Klasse der meistverbreiteten Schwachstellen dieser Art gehören Memory Corruption Schwachstellen, welche es ermöglichen
auf unbefugten Speicher eines Systems zugreifen zu können.
Genau diese Schwachstellen beispielsweise in Form eines Buffer Overflows gehören zu den meist gefundenen Schwachstellen in
Programmen, welche auf der Programmiersprache C basieren.
Aufgrund der weiten Verbreitung und gewinnenden Bedeutung von \gls{iot}-Geräten im alltäglichen Leben soll diese Arbeit
eine Möglichkeit darbieten, mithilfe von Fuzzing automatisiert Schwachstellen in diesen Geräten zu finden.
Mithilfe solcher Methodiken soll in Zukunft die Ausnutzung schwerwiegender Schwachstellen vorgebeugt werden.\\
\linebreak
Das Fuzzing von Netzwerkapplikationen ist derzeit in den Startlöchern und eine weitaus unerforschte Disziplin.
Dabei zu beachten ist, dass die Performance der Protokoll-Fuzzer weitaus unter der, der Applikations-Fuzzer liegt.
Somit bietet diese Arbeit einen alternativen Ansatz zum Fuzzen von Netzwerkprotokollen auf Applikationsebene, um diesem
Performanceverlust entgegenzuwirken.
%! Author = charon
%! Date = 2/7/24
\subsection{Zielsetzung}\label{subsec:zielsetzung}
Wie bereits erwähnt, soll diese Arbeit eine Möglichkeit offenlegen, wie ein Netzwerkprotokoll auf Applikationsebene
gefuzzt werden kann, sodass die Leistung des Fuzzers nicht darunter leidet.
Das bedeutet nicht nur, dass hier ein Weg des Protokoll-Fuzzings beschrieben wird, sondern auch die generelle Herangehensweise
an eine Fuzzing-Kampagne und die tieferen Hintergründe des \gls{afl} Fuzzers genauer erläutert werden. \\
\linebreak
Im Laufe dieser Arbeit soll die Vorgehensweise, mit der die Fuzzing-Kampagne umgesetzt wurde, dargestellt werden.
Des Weiteren werden verwendete Technologien genau erklärt und deren Rolle im Fuzzing-Prozess beleuchtet werden.
Dabei soll vor allem auf die Funktionsweise und die Interaktion von den verwendeten Technologien auf die Applikation
zum allgemeinen Verständnis der Applikation eingegangen werden.
Sobald die Technologien ausreichend erklärt wurden, folgt das Eingehen auf die genauere Implementierung und Instrumentierung
der zu untersuchenden Applikation.
Dieser Teil dient als Dokumentation zur tatsächlichen Umsetzung einer Fuzzing-Kampagne.